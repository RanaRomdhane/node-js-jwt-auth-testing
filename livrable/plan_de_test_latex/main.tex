%%%%%%%%%%%%%%%%%%%%%%%%%%%%%%%%%%%%%%%%%%%%%%%%%%%%%%%%%%%%%%%%%%%%%%%%%%%%%%%
% NODE.JS JWT AUTHENTICATION %
% PLAN DE TEST %
% CERTILOG TEMPLATE %
%%%%%%%%%%%%%%%%%%%%%%%%%%%%%%%%%%%%%%%%%%%%%%%%%%%%%%%%%%%%%%%%%%%%%%%%%%%%%%%

\documentclass[12pt,a4paper]{report}

%==============================================================================
% PACKAGES
%==============================================================================
\usepackage[utf8]{inputenc}
\usepackage[T1]{fontenc}
\usepackage[french]{babel}
\usepackage{geometry}
\usepackage{graphicx}
\usepackage{xcolor}
\usepackage{tikz}
\usepackage{tcolorbox}
\usepackage{fancyhdr}
\usepackage{titlesec}
\usepackage{titletoc}
\usepackage{hyperref}
\usepackage{longtable}
\usepackage{tabularx}
\usepackage{booktabs}
\usepackage{multirow}
\usepackage{array}
\usepackage{colortbl}
\usepackage{enumitem}
\usepackage{fontawesome5}
\usepackage{pifont}
\usepackage{calc}
\usepackage{etoolbox}
\usepackage{lastpage}
\usepackage{float}
\usepackage{wrapfig}
\usepackage{caption}
\usepackage{subcaption}
\usepackage{listings}
\usepackage{mdframed}
\usepackage{setspace}
\usepackage{amssymb}
\usepackage{amsmath}
\usepackage{pdflscape}
\usepackage{afterpage}

%==============================================================================
% TikZ LIBRARIES
%==============================================================================
\usetikzlibrary{
shapes.geometric,
arrows.meta,
positioning,
calc,
decorations.pathreplacing,
backgrounds,
fit,
shadows,
patterns,
chains,
matrix
}

%==============================================================================
% TCOLORBOX LIBRARIES
%==============================================================================
\tcbuselibrary{
skins,
breakable,
fitting,
listings
}

%==============================================================================
% PAGE GEOMETRY
%==============================================================================
\geometry{
a4paper,
left=2.5cm,
right=2.5cm,
top=3cm,
bottom=3cm,
headheight=1.5cm,
headsep=0.8cm,
footskip=1.2cm
}

%==============================================================================
% COLOR DEFINITIONS - CERTILOG INSPIRED PALETTE
%==============================================================================
% Primary Colors
\definecolor{certilogBlue}{RGB}{0, 82, 147}
\definecolor{certilogDarkBlue}{RGB}{0, 51, 102}
\definecolor{certilogLightBlue}{RGB}{0, 120, 200}
\definecolor{certilogNavy}{RGB}{25, 55, 95}

% Accent Colors
\definecolor{certilogOrange}{RGB}{243, 146, 0}
\definecolor{certilogGold}{RGB}{218, 165, 32}
\definecolor{certilogTeal}{RGB}{0, 150, 136}

% Neutral Colors
\definecolor{certilogGray}{RGB}{88, 89, 91}
\definecolor{certilogLightGray}{RGB}{241, 243, 244}
\definecolor{certilogMediumGray}{RGB}{180, 180, 180}
\definecolor{certilogDarkGray}{RGB}{60, 60, 60}

% Status Colors
\definecolor{successGreen}{RGB}{46, 125, 50}
\definecolor{warningYellow}{RGB}{255, 193, 7}
\definecolor{errorRed}{RGB}{198, 40, 40}
\definecolor{infoBlue}{RGB}{2, 136, 209}

% Background Colors
\definecolor{bgLight}{RGB}{248, 250, 252}
\definecolor{bgBlue}{RGB}{232, 245, 253}
\definecolor{bgGreen}{RGB}{232, 245, 233}
\definecolor{bgYellow}{RGB}{255, 249, 196}
\definecolor{bgRed}{RGB}{255, 235, 238}
\definecolor{bgOrange}{RGB}{255, 243, 224}

%==============================================================================
% HYPERREF CONFIGURATION
%==============================================================================
\hypersetup{
colorlinks=true,
linkcolor=certilogBlue,
filecolor=certilogBlue,
urlcolor=certilogLightBlue,
citecolor=certilogBlue,
pdftitle={Plan de Test - Authentification JWT Node.js},
pdfauthor={Oulimata SALL & Rana ROMDHANE},
pdfsubject={Test Plan TJAA-2025},
pdfkeywords={Test, JWT, Node.js, API, Authentication, Certilog},
pdfproducer={LaTeX},
pdfcreator={pdfLaTeX},
bookmarks=true,
bookmarksnumbered=true,
bookmarksopen=true,
bookmarksopenlevel=2
}

%==============================================================================
% HEADER AND FOOTER STYLES
%==============================================================================
\pagestyle{fancy}
\fancyhf{}

% Header
\fancyhead[L]{%
\begin{tikzpicture}[remember picture, overlay]
\fill[certilogBlue] (-1,-0.5) rectangle (\textwidth+1,0.8);
\node[text=white, anchor=west] at (0,0.15) {%
\small\textbf{TJAA-PT-001} | Plan de Test v1.0%
};
\node[text=white, anchor=east] at (\textwidth,0.15) {%
\small Node.js JWT Authentication Testing%
};
\end{tikzpicture}%
}

% Footer
\fancyfoot[C]{%
\begin{tikzpicture}[remember picture, overlay]
\fill[certilogLightGray] (-1,-0.3) rectangle (\textwidth+1,0.5);
\draw[certilogBlue, line width=2pt] (-1,0.5) -- (\textwidth+1,0.5);
\node[text=certilogGray, anchor=west] at (0,0.1) {%
\small\textcopyright\ 2025 ENICAR - Équipe QA%
};
\node[text=certilogBlue, anchor=center] at (0.5\textwidth,0.1) {%
\small\textbf{CONFIDENTIEL}%
};
\node[text=certilogGray, anchor=east] at (\textwidth,0.1) {%
\small Page \thepage\ sur \pageref{LastPage}%
};
\end{tikzpicture}%
}

\renewcommand{\headrulewidth}{0pt}
\renewcommand{\footrulewidth}{0pt}

% Plain style for chapter pages
\fancypagestyle{plain}{
\fancyhf{}
\fancyfoot[C]{%
\begin{tikzpicture}[remember picture, overlay]
\fill[certilogLightGray] (-1,-0.3) rectangle (\textwidth+1,0.5);
\draw[certilogBlue, line width=2pt] (-1,0.5) -- (\textwidth+1,0.5);
\node[text=certilogGray, anchor=center] at (0.5\textwidth,0.1) {%
\small Page \thepage\ sur \pageref{LastPage}%
};
\end{tikzpicture}%
}
}

%==============================================================================
% SECTION TITLE FORMATTING
%==============================================================================
% Chapter Format
\titleformat{\chapter}[display]
{\normalfont\Huge\bfseries\color{certilogDarkBlue}}
{%
\begin{tikzpicture}[remember picture, overlay]
\fill[certilogBlue] (0,0) rectangle (\textwidth,-4);
\fill[certilogOrange] (0,-4) rectangle (\textwidth,-4.2);
\node[text=white, anchor=west, font=\fontsize{60}{72}\selectfont\bfseries]
at (0.5,-2) {\thechapter};
\node[text=certilogOrange, anchor=west, font=\Large]
at (3,-2) {CHAPITRE};
\end{tikzpicture}%
}
{0pt}
{\vspace{4.5cm}\hspace{0.5cm}}
[]

\titlespacing*{\chapter}{0pt}{0pt}{40pt}

% Section Format
\titleformat{\section}
{\normalfont\Large\bfseries\color{certilogBlue}}
{\colorbox{certilogBlue}{\textcolor{white}{\thesection}}}
{1em}
{}
[{\titlerule[2pt]}]

\titlespacing*{\section}{0pt}{20pt}{10pt}

% Subsection Format
\titleformat{\subsection}
{\normalfont\large\bfseries\color{certilogDarkBlue}}
{\textcolor{certilogOrange}{\thesubsection}}
{0.8em}
{}

\titlespacing*{\subsection}{0pt}{15pt}{8pt}

% Subsubsection Format
\titleformat{\subsubsection}
{\normalfont\normalsize\bfseries\color{certilogNavy}}
{\textcolor{certilogTeal}{\thesubsubsection}}
{0.6em}
{}

\titlespacing*{\subsubsection}{0pt}{10pt}{5pt}

%==============================================================================
% CUSTOM TCOLORBOX ENVIRONMENTS
%==============================================================================

% Document Info Box
\newtcolorbox{docinfo}[1][]{
enhanced,
colback=bgBlue,
colframe=certilogBlue,
coltitle=white,
fonttitle=\bfseries\large,
title=#1,
boxrule=1pt,
arc=3mm,
left=10pt,
right=10pt,
top=8pt,
bottom=8pt,
shadow={2mm}{-2mm}{0mm}{black!20},
breakable
}

% Important Note Box
\newtcolorbox{notebox}[1][Note]{
enhanced,
colback=bgYellow,
colframe=warningYellow,
coltitle=certilogDarkGray,
fonttitle=\bfseries,
title={\faExclamationTriangle\ #1},
boxrule=0pt,
leftrule=4pt,
arc=0mm,
left=10pt,
right=10pt,
top=5pt,
bottom=5pt,
breakable
}

% Success Box
\newtcolorbox{successbox}[1][Succès]{
enhanced,
colback=bgGreen,
colframe=successGreen,
coltitle=white,
fonttitle=\bfseries,
title={\faCheckCircle\ #1},
boxrule=0pt,
leftrule=4pt,
arc=0mm,
left=10pt,
right=10pt,
top=5pt,
bottom=5pt,
breakable
}

% Error Box
\newtcolorbox{errorbox}[1][Attention]{
enhanced,
colback=bgRed,
colframe=errorRed,
coltitle=white,
fonttitle=\bfseries,
title={\faTimesCircle\ #1},
boxrule=0pt,
leftrule=4pt,
arc=0mm,
left=10pt,
right=10pt,
top=5pt,
bottom=5pt,
breakable
}

% Info Box
\newtcolorbox{infobox}[1][Information]{
enhanced,
colback=bgBlue,
colframe=infoBlue,
coltitle=white,
fonttitle=\bfseries,
title={\faInfoCircle\ #1},
boxrule=0pt,
leftrule=4pt,
arc=0mm,
left=10pt,
right=10pt,
top=5pt,
bottom=5pt,
breakable
}

% Metric Box
\newtcolorbox{metricbox}[1][]{
enhanced,
colback=certilogLightGray,
colframe=certilogBlue,
coltitle=white,
fonttitle=\bfseries,
title=#1,
boxrule=1pt,
arc=2mm,
left=8pt,
right=8pt,
top=5pt,
bottom=5pt,
breakable
}

% Code Box
\newtcolorbox{codebox}[1][Code]{
enhanced,
colback=certilogDarkGray,
colframe=certilogDarkGray,
coltitle=certilogOrange,
fonttitle=\bfseries\ttfamily,
title=#1,
boxrule=0pt,
arc=3mm,
left=10pt,
right=10pt,
top=8pt,
bottom=8pt,
breakable
}

%==============================================================================
% CUSTOM LIST STYLES
%==============================================================================
\setlist[itemize,1]{
label=\textcolor{certilogBlue}{\faChevronRight},
leftmargin=2em,
itemsep=3pt
}

\setlist[itemize,2]{
label=\textcolor{certilogOrange}{\faAngleRight},
leftmargin=1.5em,
itemsep=2pt
}

\setlist[enumerate,1]{
label=\textcolor{certilogBlue}{\arabic*.},
leftmargin=2em,
itemsep=3pt
}

% Checkmark list
\newlist{checklist}{itemize}{2}
\setlist[checklist]{
label=\textcolor{successGreen}{\faCheckSquare},
leftmargin=2em,
itemsep=3pt
}

%==============================================================================
% CUSTOM TABLE COLUMN TYPES
%==============================================================================
\newcolumntype{L}[1]{>{\raggedright\arraybackslash}p{#1}}
\newcolumntype{C}[1]{>{\centering\arraybackslash}p{#1}}
\newcolumntype{R}[1]{>{\raggedleft\arraybackslash}p{#1}}

% Table header style
\newcommand{\tableheader}[1]{\textbf{\textcolor{white}{#1}}}
\newcommand{\headerrow}{\rowcolor{certilogBlue}}

%==============================================================================
% CUSTOM COMMANDS
%==============================================================================

% Version badge
\newcommand{\versionbadge}[1]{%
\tikz[baseline=(badge.base)]{
\node[fill=certilogBlue, text=white, rounded corners=2pt,
inner xsep=5pt, inner ysep=2pt, font=\small\bfseries] (badge) {#1};
}%
}

% Status badge
\newcommand{\statusbadge}[2]{%
\tikz[baseline=(badge.base)]{
\node[fill=#2, text=white, rounded corners=2pt,
inner xsep=5pt, inner ysep=2pt, font=\small\bfseries] (badge) {#1};
}%
}

% Priority badge
\newcommand{\prioritybadge}[1]{%
\ifcase#1\or
\statusbadge{P1}{errorRed}%
\or
\statusbadge{P2}{certilogOrange}%
\or
\statusbadge{P3}{warningYellow}%
\or
\statusbadge{P4}{certilogMediumGray}%
\fi
}

% Criticality indicator
\newcommand{\critical}{\textcolor{errorRed}{\faBolt\ \textbf{CRITIQUE}}}
\newcommand{\high}{\textcolor{certilogOrange}{\faExclamationTriangle\ \textbf{HAUTE}}}
\newcommand{\medium}{\textcolor{warningYellow}{\faInfoCircle\ \textbf{MOYENNE}}}
\newcommand{\low}{\textcolor{certilogGray}{\faArrowDown\ \textbf{FAIBLE}}}

% Checkmarks and crosses
\newcommand{\yes}{\textcolor{successGreen}{\faCheck}}
\newcommand{\no}{\textcolor{errorRed}{\faTimes}}
\newcommand{\partial}{\textcolor{warningYellow}{\faAdjust}}

% Probability indicators
\newcommand{\probhigh}{\textcolor{errorRed}{\faCircle\ Élevée}}
\newcommand{\probmedium}{\textcolor{warningYellow}{\faCircle\ Moyenne}}
\newcommand{\problow}{\textcolor{successGreen}{\faCircle\ Faible}}

%==============================================================================
% LISTINGS CONFIGURATION
%==============================================================================
\lstset{
basicstyle=\ttfamily\small\color{white},
backgroundcolor=\color{certilogDarkGray},
keywordstyle=\color{certilogOrange}\bfseries,
stringstyle=\color{successGreen},
commentstyle=\color{certilogMediumGray}\itshape,
numbers=left,
numberstyle=\tiny\color{certilogMediumGray},
numbersep=8pt,
frame=none,
breaklines=true,
breakatwhitespace=true,
tabsize=2,
showstringspaces=false,
xleftmargin=15pt,
xrightmargin=10pt
}

%==============================================================================
% DOCUMENT INFORMATION
%==============================================================================
\newcommand{\docTitle}{Plan de Test}
\newcommand{\docSubtitle}{Authentification JWT Node.js}
\newcommand{\docRef}{TJAA-PT-001}
\newcommand{\docVersion}{v1.0}
\newcommand{\projectRef}{TJAA-2025}
\newcommand{\docDate}{01/12/2025}
\newcommand{\authorOne}{Oulimata SALL}
\newcommand{\authorTwo}{Rana ROMDHANE}

%==============================================================================
% DOCUMENT START
%==============================================================================
\begin{document}

%==============================================================================
% COVER PAGE
%==============================================================================
\begin{titlepage}
\thispagestyle{empty}

% Background decoration
\begin{tikzpicture}[remember picture, overlay]
% Top banner
\fill[certilogBlue]
(current page.north west) rectangle
([yshift=-6cm]current page.north east);

text

% Orange accent line
\fill[certilogOrange] 
    ([yshift=-6cm]current page.north west) rectangle 
    ([yshift=-6.3cm]current page.north east);

% Bottom banner
\fill[certilogDarkBlue] 
    (current page.south west) rectangle 
    ([yshift=3cm]current page.south east);

% Decorative circles
\foreach \x/\y/\r/\o in {-2/2/3/0.1, 18/1/4/0.08, 16/-2/2/0.15} {
    \fill[white, opacity=\o] ([xshift=\x cm, yshift=\y cm]current page.north west) 
        circle (\r cm);
}

% Grid pattern in header
\foreach \i in {0,0.5,...,21} {
    \draw[white, opacity=0.05, line width=0.5pt] 
        ([xshift=\i cm]current page.north west) -- 
        ([xshift=\i cm, yshift=-6cm]current page.north west);
}

% Logo placeholder (CERTILOG style)
\node[anchor=north west, inner sep=0] at ([xshift=2cm, yshift=-1.5cm]current page.north west) {
    \begin{tikzpicture}
        \node[fill=white, rounded corners=5pt, inner sep=10pt, 
              drop shadow={shadow xshift=2pt, shadow yshift=-2pt, opacity=0.3}] {
            \begin{tabular}{l}
                \textcolor{certilogBlue}{\fontsize{24}{28}\selectfont\bfseries CERTILOG}\\[-2pt]
                \textcolor{certilogOrange}{\small\itshape Template}
            \end{tabular}
        };
    \end{tikzpicture}
};

% Document type badge
\node[anchor=north east, inner sep=0] at ([xshift=-2cm, yshift=-2cm]current page.north east) {
    \begin{tikzpicture}
        \node[fill=certilogOrange, text=white, rounded corners=3pt, 
              inner xsep=15pt, inner ysep=8pt, font=\large\bfseries] {
            PLAN DE TEST
        };
    \end{tikzpicture}
};

% Main title area
\node[anchor=center, inner sep=0] at ([yshift=-1cm]current page.center) {
    \begin{tikzpicture}
        % Title box
        \node[fill=white, rounded corners=10pt, inner sep=25pt,
              drop shadow={shadow xshift=3pt, shadow yshift=-3pt, opacity=0.2},
              minimum width=14cm] (titlebox) {
            \begin{minipage}{13cm}
                \centering
                
                % Icon
                \begin{tikzpicture}
                    \node[circle, fill=certilogBlue, minimum size=1.5cm] {
                        \textcolor{white}{\fontsize{24}{28}\selectfont\faLock}
                    };
                \end{tikzpicture}
                
                \vspace{15pt}
                
                % Title
                {\fontsize{28}{34}\selectfont\bfseries\color{certilogDarkBlue}
                NODE.JS JWT\\[5pt]
                AUTHENTICATION TESTING}
                
                \vspace{20pt}
                
                % Divider
                \begin{tikzpicture}
                    \draw[certilogOrange, line width=3pt] (0,0) -- (8,0);
                    \fill[certilogBlue] (4,0) circle (5pt);
                \end{tikzpicture}
                
                \vspace{20pt}
                
                % Project reference
                {\Large\color{certilogGray} Projet: \textbf{\projectRef}}
                
                \vspace{10pt}
                
                % Document reference
                {\large\color{certilogGray} Référence: \textbf{\docRef}}
                
            \end{minipage}
        };
    \end{tikzpicture}
};

% Version and date info
\node[anchor=south, inner sep=0] at ([yshift=6cm]current page.south) {
    \begin{tikzpicture}
        \node[fill=certilogLightGray, rounded corners=5pt, inner sep=15pt] {
            \begin{tabular}{ccc}
                \textcolor{certilogBlue}{\faCalendarAlt} & 
                \textcolor{certilogBlue}{\faCodeBranch} & 
                \textcolor{certilogBlue}{\faUsers} \\[5pt]
                \textbf{\docDate} & 
                \textbf{\docVersion} & 
                \textbf{Équipe QA}
            \end{tabular}
        };
    \end{tikzpicture}
};

% Authors
\node[anchor=south, text=white, inner sep=0] at ([yshift=1cm]current page.south) {
    \begin{tabular}{c}
        {\large\bfseries Préparé par}\\[8pt]
        {\Large \authorOne\ \textcolor{certilogOrange}{\&}\ \authorTwo}\\[5pt]
        {\small ENICAR - École Nationale d'Ingénieurs de Carthage}
    \end{tabular}
};
\end{tikzpicture}

\end{titlepage}

%==============================================================================
% DOCUMENT IDENTIFICATION PAGE
%==============================================================================
\newpage
\thispagestyle{empty}

\vspace*{1cm}

\begin{center}
{\Huge\bfseries\color{certilogBlue} IDENTIFICATION DU DOCUMENT}
\end{center}

\vspace{1cm}

\begin{docinfo}[Informations Générales]
\begin{center}
\renewcommand{\arraystretch}{1.5}
\begin{tabular}{|>{\columncolor{certilogLightGray}}L{5cm}|L{9cm}|}
\hline
\textbf{Nom du document} & Plan de Test - Authentification JWT Node.js \
\hline
\textbf{Référence du projet} & TJAA-2025 \
\hline
\textbf{Référence du document} & TJAA-PT-001 \
\hline
\textbf{Version} & v 1.0 \
\hline
\textbf{Préparé par} & Oulimata SALL & Rana ROMDHANE \
\hline
\textbf{Date} & 01/12/2025 \
\hline
\end{tabular}
\end{center}
\end{docinfo}

\vspace{1cm}

\begin{docinfo}[Historique des Changements]
\begin{center}
\renewcommand{\arraystretch}{1.4}
\begin{tabular}{|C{1.2cm}|C{2.5cm}|C{2.5cm}|L{3cm}|L{4.5cm}|}
\hline
\headerrow
\tableheader{Version} & \tableheader{ID Demande} & \tableheader{Date} & \tableheader{Modifié par} & \tableheader{Description} \
\hline
1.0 & TJAA-PT-001 & 01/12/2025 & O. SALL & R. ROMDHANE & Création initiale du plan de test \
\hline
\end{tabular}
\end{center}
\end{docinfo}

\vspace{1cm}

\begin{docinfo}[Distribution et Validation]

\begin{infobox}[Méthode RACI]
\begin{itemize}[leftmargin=1cm]
\item \textbf{R} : Réalisation (Responsible)
\item \textbf{A} : Approbation (Accountable)
\item \textbf{C} : Consultation (Consulted)
\item \textbf{I} : Information (Informed)
\end{itemize}
\end{infobox}

\vspace{0.5cm}

\begin{center}
\renewcommand{\arraystretch}{1.4}
\begin{tabular}{|L{3.5cm}|L{2.5cm}|C{2cm}|C{1.2cm}|C{2.8cm}|}
\hline
\headerrow
\tableheader{NOM Prénom} & \tableheader{Rôle} & \tableheader{Entité} & \tableheader{RACI} & \tableheader{Date validation} \
\hline
SALL Oulimata & Test Lead & Équipe QA & R & 01/12/2025 \
\hline
ROMDHANE Rana & Test Engineer & Équipe QA & R & 01/12/2025 \
\hline
AOUADI Hela & Encadrant & ENICAR & A & À valider \
\hline
\end{tabular}
\end{center}
\end{docinfo}

%==============================================================================
% TABLE OF CONTENTS
%==============================================================================
\newpage
\thispagestyle{plain}

\begin{tikzpicture}[remember picture, overlay]
\fill[certilogBlue] (current page.north west) rectangle ([yshift=-3.5cm]current page.north east);
\node[anchor=west, text=white, font=\Huge\bfseries] at ([xshift=2cm, yshift=-2cm]current page.north west) {
\faListOl\quad TABLE DES MATIÈRES
};
\end{tikzpicture}

\vspace{3cm}

\setcounter{tocdepth}{3}

% Customize TOC appearance
\titlecontents{chapter}[0pt]
{\addvspace{15pt}\bfseries\large\color{certilogBlue}}
{\thecontentslabel\quad}
{}
{\hfill\contentspage}

\titlecontents{section}[20pt]
{\addvspace{5pt}\color{certilogDarkBlue}}
{\thecontentslabel\quad}
{}
{\titlerule*[8pt]{.}\contentspage}

\titlecontents{subsection}[40pt]
{\small\color{certilogGray}}
{\thecontentslabel\quad}
{}
{\titlerule*[8pt]{.}\contentspage}

\tableofcontents

%==============================================================================
% CHAPTER 1: INTRODUCTION
%==============================================================================
\chapter{INTRODUCTION}

\section{Objectif}

Ce Plan de Test est élaboré conformément au template \textbf{Certilog} et constitue le document officiel définissant l'ensemble des activités de tests pour le projet \textbf{Node.js JWT Authentication Testing}, développé par \textbf{Oulimata SALL} et \textbf{Rana ROMDHANE}.

\begin{successbox}[Objectifs Principaux]
\begin{checklist}
\item Vérifier la conformité fonctionnelle de l'API d'authentification JWT
\item Valider les aspects techniques (middleware, base de données MongoDB)
\item Assurer la robustesse des aspects non fonctionnels (sécurité, performance)
\item Garantir la fiabilité du pipeline CI/CD (GitHub Actions)
\item Confirmer la maintenabilité et la qualité du code
\end{checklist}
\end{successbox}

\vspace{0.5cm}

Ce document sert de référence pour toutes les parties prenantes du projet et définit clairement les stratégies, ressources, livrables et critères de qualité attendus.

\section{Références}

\begin{center}
\renewcommand{\arraystretch}{1.4}
\begin{tabular}{|C{1.2cm}|L{4.5cm}|L{6.5cm}|C{2cm}|}
\hline
\headerrow
\tableheader{ID} & \tableheader{Titre du document} & \tableheader{Lien ou emplacement} & \tableheader{Date} \
\hline
REF-01 & Dépôt GitHub du projet & \url{https://github.com/RanaRomdhane/node-js-jwt-auth-testing} & 15/11/2025 \
\hline
REF-02 & Campagne de tests Jira/Xray & {[TJAA-35] Campagne de Test - Sprint 1} & 20/11/2025 \
\hline
REF-03 & Modèle Certilog & Certilog_Plan_de_Test_1.0.1 & 01/10/2025 \
\hline
REF-04 & Documentation API & README.md du repository GitHub & 15/11/2025 \
\hline
\end{tabular}
\end{center}

%==============================================================================
% CHAPTER 2: APERÇU GÉNÉRAL DU PROJET
%==============================================================================
\chapter{APERÇU GÉNÉRAL DU PROJET}

Le projet consiste à concevoir, développer, tester et automatiser une API RESTful moderne utilisant les technologies suivantes :

\begin{center}
\begin{tikzpicture}[
node distance=2.5cm,
tech/.style={
rectangle, rounded corners=8pt,
minimum width=3cm, minimum height=1.2cm,
draw=certilogBlue, line width=1.5pt,
fill=bgBlue, font=\bfseries,
drop shadow
}
]
\node[tech] (nodejs) {\faNodeJs\ Node.js};
\node[tech, right=of nodejs] (mongodb) {\faDatabase\ MongoDB};
\node[tech, right=of mongodb] (jwt) {\faKey\ JWT};

text

\draw[-{Stealth[length=3mm]}, certilogOrange, line width=2pt] 
    (nodejs) -- (mongodb);
\draw[-{Stealth[length=3mm]}, certilogOrange, line width=2pt] 
    (mongodb) -- (jwt);
\end{tikzpicture}
\end{center}

\subsection{Fonctionnalités Principales}

\begin{metricbox}[Fonctionnalités Métier]
\begin{itemize}
\item Gestion complète de l'authentification utilisateur (inscription, connexion, déconnexion)
\item Système de gestion des rôles et permissions (user, moderator, admin)
\item Sécurisation des endpoints via JWT
\item Accès conditionnel aux ressources protégées selon les rôles
\item Validation robuste des données entrantes
\end{itemize}
\end{metricbox}

\subsection{Périmètre des Tests}

\begin{center}
\begin{tikzpicture}[
scope/.style={
rectangle, rounded corners=5pt,
minimum width=4.5cm, minimum height=1cm,
font=\small\bfseries, text=white
}
]
% Test categories
\node[scope, fill=certilogBlue] (func) at (0,0) {Tests Fonctionnels};
\node[scope, fill=certilogTeal] (tech) at (5,0) {Tests Techniques};
\node[scope, fill=certilogOrange] (sec) at (10,0) {Tests Sécurité};
\node[scope, fill=successGreen] (perf) at (2.5,-1.5) {Tests Performance};
\node[scope, fill=certilogNavy] (maint) at (7.5,-1.5) {Tests Maintenabilité};

text

% Central node
\node[circle, fill=certilogDarkBlue, text=white, minimum size=2cm, font=\bfseries] 
    (center) at (5,-3.5) {CI/CD};

% Connections
\foreach \source in {func, tech, sec, perf, maint} {
    \draw[certilogGray, line width=1pt, dashed] (\source) -- (center);
}
\end{tikzpicture}
\end{center}

\section{Jalons du Projet}

\begin{center}
\renewcommand{\arraystretch}{1.4}
\begin{tabular}{|L{5cm}|C{2.5cm}|L{6cm}|}
\hline
\headerrow
\tableheader{Jalons métiers} & \tableheader{Date} & \tableheader{Commentaires} \
\hline
Étude d'opportunité (T-1) & 01/10/2025 & Analyse de faisabilité réalisée par l'entreprise \
\hline
Analyse détaillée (T0) & 05/10/2025 & Étude approfondie du cahier des charges \
\hline
Développement (T1) & 15/10/2025 & Mise en œuvre de l'API et des fonctionnalités \
\hline
Tests et validation (T1.5) & 01/11/2025 & Phase de tests complète (objet de ce document) \
\hline
Déploiement (T2) & 26/11/2025 & Mise en production de l'application \
\hline
Lancement sur le marché (T3) & 28/11/2026 & Diffusion et communication externe \
\hline
Clôture du projet (T4) & 29/11/2026 & Bilan final et documentation de clôture \
\hline
\end{tabular}
\end{center}

\section{Jalons Clés du Projet et du Test}

\begin{center}
\renewcommand{\arraystretch}{1.4}
\begin{tabular}{|L{4.5cm}|C{3cm}|L{5.5cm}|}
\hline
\headerrow
\tableheader{Jalons clés du test} & \tableheader{Date} & \tableheader{Commentaires} \
\hline
Livraison des endpoints REST & 01/10/2025 & Début officiel du projet de test \
\hline
Mise en place du pipeline CI/CD & 17/11/2025 & Fondation de l'automatisation \
\hline
Création des cas de tests & 20/10 - 25/10/2025 & Rédaction des scénarios dans Xray/Jira \
\hline
Exécution des tests unitaires & 01/11/2025 & Validation des fonctions isolées (Jest) \
\hline
Exécution des tests d'intégration & 02/11/2025 & Tests API avec Supertest \
\hline
Exécution des tests E2E & 02/11/2025 & Scénarios utilisateur (Chai/Postman) \
\hline
Tests de sécurité & 05/11/2025 & Validation des vulnérabilités \
\hline
Tests de performance & 05/11/2025 & Mesure des temps de réponse \
\hline
Analyse des résultats & 20/11/2025 & Consolidation des métriques \
\hline
Publication des rapports & 01/12/2025 & Rapports de couverture et qualité \
\hline
Validation finale & 15/12/2025 & Acceptation du projet de test \
\hline
\end{tabular}
\end{center}

%==============================================================================
% CHAPTER 3: ÉLÉMENTS À TESTER
%==============================================================================
\chapter{ÉLÉMENTS À TESTER}

\begin{center}
\renewcommand{\arraystretch}{1.4}
\begin{tabular}{|C{2cm}|L{3.5cm}|C{1.5cm}|C{2cm}|L{4cm}|}
\hline
\headerrow
\tableheader{ID} & \tableheader{Nom} & \tableheader{Version} & \tableheader{Type} & \tableheader{Commentaires} \
\hline
API-AUTH-01 & Code source Node.js & 1.0 & Logiciel & Logique métier de l'API \
\hline
API-AUTH-02 & Middleware JWT & 1.0 & Logiciel & Gestion authentification \
\hline
API-AUTH-03 & Modèles MongoDB & 1.0 & Logiciel & Schémas User, Role \
\hline
API-AUTH-04 & Pipeline CI/CD & 1.0 & Logiciel & GitHub Actions \
\hline
API-AUTH-05 & Documentation & 1.0 & Document & README, API docs \
\hline
API-AUTH-06 & Tests automatisés & 1.0 & Logiciel & Jest + Supertest \
\hline
API-AUTH-07 & Configuration & 1.0 & Config & Variables d'environnement \
\hline
\end{tabular}
\end{center}

%==============================================================================
% CHAPTER 4: CARACTÉRISTIQUES À TESTER
%==============================================================================
\chapter{CARACTÉRISTIQUES À TESTER}

Les caractéristiques suivantes seront testées de manière exhaustive :

\subsection{Fonctionnalités Métier}

\begin{checklist}
\item Inscription utilisateur (signup) avec validation des données
\item Connexion utilisateur (signin) et génération de tokens JWT
\item Gestion des rôles (user, moderator, admin)
\item Accès protégé aux endpoints selon les rôles
\item Refresh token et renouvellement de session
\end{checklist}

\subsection{Aspects Techniques}

\begin{checklist}
\item Validation des données d'entrée (format, longueur, types)
\item Hachage sécurisé des mots de passe (bcrypt)
\item Génération et vérification des tokens JWT
\item Middleware d'authentification et autorisation
\item Intégration MongoDB (CRUD opérations)
\end{checklist}

\subsection{Aspects Non Fonctionnels}

\begin{checklist}
\item Sécurité (injections SQL/NoSQL, XSS, tokens invalides/expirés)
\item Performance (temps de réponse, charge modérée)
\item Maintenabilité du code (qualité, lisibilité, respect des standards)
\end{checklist}

\subsection{Infrastructure}

\begin{checklist}
\item Tests automatisés dans le pipeline CI/CD
\item Couverture de code (statements, branches, functions)
\item Analyse statique du code (ESLint, SonarQube)
\end{checklist}

%==============================================================================
% CHAPTER 5: CARACTÉRISTIQUES À NE PAS TESTER
%==============================================================================
\chapter{CARACTÉRISTIQUES À NE PAS TESTER}

Certaines fonctionnalités sont explicitement hors périmètre pour les raisons suivantes :

\begin{center}
\renewcommand{\arraystretch}{1.4}
\begin{tabular}{|L{4.5cm}|L{9cm}|}
\hline
\headerrow
\tableheader{Caractéristique} & \tableheader{Raison de l'exclusion} \
\hline
Interface front-end (UI) & Aucune interface utilisateur n'existe (API backend uniquement) \
\hline
Tests de charge extrême & Stress tests > 10 000 req/s non requis pour ce projet académique \
\hline
Scalabilité horizontale & Clusterisation et load balancing non prévus \
\hline
Tests multi-navigateurs & Pas d'interface web à tester \
\hline
Compatibilité mobile & Application backend uniquement \
\hline
Intégrations tierces & Aucun service externe (paiement, email) \
\hline
Tests de récupération & Backup/restore non implémentés \
\hline
Tests d'accessibilité & Non applicable pour une API REST \
\hline
\end{tabular}
\end{center}

\begin{notebox}[Note importante]
Ces exclusions pourront être reconsidérées dans les versions futures du projet selon l'évolution des besoins.
\end{notebox}

%==============================================================================
% CHAPTER 6: APPROCHE TEST
%==============================================================================
\chapter{APPROCHE TEST}

L'approche de test adoptée repose sur une méthodologie hybride combinant :

\begin{itemize}
\item Tests fonctionnels et techniques
\item Tests automatisés (majoritaires) et manuels (ponctuels)
\item Tests à différents niveaux (unitaires, intégration, système)
\item Traçabilité complète des exigences aux résultats
\end{itemize}

\begin{infobox}[Principes Directeurs]
\begin{description}[leftmargin=2cm, labelwidth=1.8cm]
\item[\textbf{Automatisation maximale}] Priorité aux tests automatisés pour garantir la répétabilité
\item[\textbf{Shift-left testing}] Tests précoces dès le développement
\item[\textbf{Continuous testing}] Intégration dans le pipeline CI/CD
\item[\textbf{Traçabilité totale}] Liens exigences 
↔
↔ tests 
↔
↔ défauts
\item[\textbf{Qualité mesurable}] Métriques objectives et KPI définis
\end{description}
\end{infobox}

\section{Criticité des Caractéristiques à Tester}

La criticité est évaluée selon l'impact métier et le risque technique :

\begin{center}
\renewcommand{\arraystretch}{1.4}
\begin{tabular}{|L{4.5cm}|C{2.5cm}|L{6cm}|}
\hline
\headerrow
\tableheader{Caractéristique} & \tableheader{Criticité} & \tableheader{Justification} \
\hline
Inscription / Connexion & \critical & Fonctionnalité fondamentale bloquante \
\hline
Sécurité JWT & \critical & Faille = compromission totale \
\hline
Gestion des rôles & \high & Contrôle d'accès essentiel \
\hline
Validation des données & \high & Protection contre injections \
\hline
Pipeline CI/CD & \high & Garantit la qualité en continu \
\hline
Performance API & \medium & Important mais non bloquant \
\hline
Documentation & \low & Utile mais non critique \
\hline
\end{tabular}
\end{center}

\section{Effort de Test}

L'effort de test est distribué stratégiquement sur les différents niveaux :

\begin{center}
\begin{tikzpicture}
% Pie chart
\foreach \p/\c/\l [count=\i from 0] in {
30/certilogBlue/Tests unitaires,
25/certilogTeal/Tests intégration,
25/certilogOrange/Tests E2E,
10/errorRed/Tests sécurité,
5/successGreen/Tests performance,
5/certilogNavy/CI/CD
} {
\pgfmathsetmacro{\startangle}{90 + \i60}
\pgfmathsetmacro{\endangle}{\startangle - \p3.6}

text

    \fill[\c] (0,0) -- (\startangle:2.5) arc (\startangle:\endangle:2.5) -- cycle;
    
    \pgfmathsetmacro{\midangle}{(\startangle + \endangle)/2}
    \node[font=\small\bfseries, text=white] at (\midangle:1.8) {\p\%};
}

% Legend
\foreach \c/\l [count=\i from 0] in {
    certilogBlue/Tests unitaires (30\%),
    certilogTeal/Tests intégration (25\%),
    certilogOrange/Tests E2E (25\%),
    errorRed/Tests sécurité (10\%),
    successGreen/Tests performance (5\%),
    certilogNavy/Vérification CI/CD (5\%)
} {
    \fill[\c] (5, 2.5 - \i*0.6) rectangle ++(0.4, 0.4);
    \node[anchor=west, font=\small] at (5.6, 2.7 - \i*0.6) {\l};
}
\end{tikzpicture}
\end{center}

\vspace{0.5cm}

\textbf{Total estimé :} 8 jours/personne pour l'ensemble des activités de test.

\section{Niveaux de Test}

\begin{metricbox}[Niveau 1 - Tests Unitaires (Jest)]
\begin{itemize}
\item \textbf{Objectif :} Valider les fonctions isolées
\item \textbf{Scope :} Fonctions utilitaires, validation, hashing, génération JWT
\item \textbf{Outils :} Jest avec mocks et stubs
\item \textbf{Critère de succès :} 100% des fonctions critiques testées
\end{itemize}
\end{metricbox}

\begin{metricbox}[Niveau 2 - Tests d'Intégration (Supertest + MongoDB)]
\begin{itemize}
\item \textbf{Objectif :} Vérifier les interactions API 
↔
↔ Base de données
\item \textbf{Scope :} Endpoints REST, middleware, persistance MongoDB
\item \textbf{Outils :} Supertest, MongoDB Memory Server
\item \textbf{Critère de succès :} Tous les endpoints testés
\end{itemize}
\end{metricbox}

\begin{metricbox}[Niveau 3 - Tests Système / E2E (Chai / Postman)]
\begin{itemize}
\item \textbf{Objectif :} Valider les parcours utilisateur complets
\item \textbf{Scope :} Scénarios métier de bout en bout
\item \textbf{Outils :} Chai, Postman/Newman
\item \textbf{Critère de succès :} Tous les parcours critiques validés
\end{itemize}
\end{metricbox}

\begin{metricbox}[Niveau 4 - Tests Non Fonctionnels]
\begin{itemize}
\item \textbf{Objectif :} Valider sécurité et performance
\item \textbf{Scope :} Vulnérabilités OWASP, temps de réponse, charge
\item \textbf{Outils :} Scripts personnalisés, autocannon/loadtest
\item \textbf{Critère de succès :} Conformité aux standards
\end{itemize}
\end{metricbox}

\begin{metricbox}[Niveau 5 - Tests CI/CD (GitHub Actions)]
\begin{itemize}
\item \textbf{Objectif :} Garantir la stabilité du pipeline
\item \textbf{Scope :} Installation, lint, tests, build, déploiement
\item \textbf{Outils :} GitHub Actions, rapports automatiques
\item \textbf{Critère de succès :} Pipeline stable avec 95%+ de succès
\end{itemize}
\end{metricbox}

\section{Techniques de Test}

\subsection{Tests Statiques}

\begin{itemize}
\item \textbf{ESLint :} Analyse syntaxique et respect des conventions
\item \textbf{SonarQube :} Détection de code smells, bugs potentiels, vulnérabilités
\item \textbf{Audit manuel :} Revue de code peer-to-peer
\end{itemize}

\subsection{Tests Dynamiques}

\begin{itemize}
\item \textbf{Jest :} Exécution des tests unitaires avec assertions
\item \textbf{Supertest :} Tests HTTP des endpoints API
\item \textbf{Chai :} Assertions avancées pour tests E2E
\end{itemize}

\subsection{Techniques de Conception de Tests}

\begin{itemize}
\item \textbf{Partitionnement en classes d'équivalence :} Validation des formulaires (email valide/invalide)
\item \textbf{Analyse des valeurs limites :} Longueur minimale/maximale des champs
\item \textbf{Tests en boîte noire :} Validation des entrées/sorties API
\item \textbf{Tests en boîte blanche :} Couverture du code, branches conditionnelles
\end{itemize}

\section{Priorisation de l'Exécution des Tests}

\subsection{Niveau 1 - Tests Unitaires}

\textbf{Objectif :} Valider les fonctions internes de l'API.

\begin{infobox}[Critères de Priorisation]
\begin{description}
\item[\faCode\ Complexité du code] Fonctions avec logique conditionnelle avancée
\item[\faBolt\ Risque technique] Fonctions susceptibles de provoquer des erreurs silencieuses
\item[\faProjectDiagram\ Impact sur composants] Fonctions réutilisées dans plusieurs modules
\item[\faSync\ Fréquence d'appel] Fonctions fortement sollicitées
\end{description}
\end{infobox}

\textbf{Ordre d'exécution :}
\begin{enumerate}
\item Fonctions de validation des données
\item Hashing et cryptographie (bcrypt, JWT)
\item Utilitaires et helpers
\item Fonctions métier simples
\end{enumerate}

\subsection{Niveau 2 - Tests d'Intégration}

\textbf{Objectif :} Vérifier l'interaction API 
↔
↔ Base de données / middleware.

\begin{infobox}[Critères de Priorisation]
\begin{description}
\item[\faBriefcase\ Risque métier] Scénarios d'inscription et connexion
\item[\faExclamationTriangle\ Sévérité potentielle] Erreurs d'intégration bloquantes
\item[\faExchangeAlt\ Criticité des flux] Login 
→
→ génération JWT 
→
→ accès protégé
\item[\faChartLine\ Fréquence d'utilisation] Endpoints les plus appelés
\end{description}
\end{infobox}

\textbf{Ordre d'exécution :}
\begin{enumerate}
\item Endpoints d'authentification (signup, signin)
\item Endpoints protégés par rôles
\item Middleware d'autorisation
\item Opérations CRUD MongoDB
\item Endpoints secondaires
\end{enumerate}

\subsection{Niveau 3 - Tests Système / E2E}

\textbf{Objectif :} Valider le parcours utilisateur complet.

\textbf{Ordre d'exécution :}
\begin{enumerate}
\item Parcours utilisateur standard (signup 
→
→ login 
→
→ accès)
\item Parcours avec différents rôles (user, moderator, admin)
\item Scénarios de gestion de session (refresh, logout)
\item Scénarios d'erreur (credentials incorrects, accès refusé)
\end{enumerate}

\subsection{Niveau 4 - Tests Non Fonctionnels}

\subsubsection{Performance}

Tests prioritaires :
\begin{itemize}
\item Temps de réponse endpoints authentification (< 200ms)
\item Comportement sous charge modérée (50 req/s)
\item Stabilité mémoire et CPU
\end{itemize}

\subsubsection{Sécurité}

Tests prioritaires :
\begin{itemize}
\item Tokens invalides/expirés/manipulés
\item Tentatives d'injection NoSQL
\item Brute force sur login (rate limiting)
\item Accès aux endpoints sans authentification
\item Élévation de privilèges (user 
→
→ admin)
\end{itemize}

\section{Automatisation des Tests}

\begin{center}
\renewcommand{\arraystretch}{1.4}
\begin{tabular}{|L{3.5cm}|C{2cm}|L{3.5cm}|C{3cm}|}
\hline
\headerrow
\tableheader{Niveau de test} & \tableheader{Auto.} & \tableheader{Outil principal} & \tableheader{Fréquence} \
\hline
Tests unitaires & \yes\ 100% & Jest & Chaque commit \
\hline
Tests d'intégration & \yes\ 100% & Supertest & Chaque commit \
\hline
Tests E2E & \yes\ 90% & Chai / Newman & Chaque push main \
\hline
Tests de sécurité & \yes\ 80% & Scripts custom & Quotidien \
\hline
Tests de performance & \yes\ 70% & loadtest/autocannon & Hebdomadaire \
\hline
Linting & \yes\ 100% & ESLint & Chaque commit \
\hline
Analyse de couverture & \yes\ 100% & Jest coverage & Chaque push \
\hline
Pipeline CI/CD & \yes\ 100% & GitHub Actions & Chaque push \
\hline
\end{tabular}
\end{center}

\begin{successbox}[Bénéfices de l'Automatisation]
\begin{itemize}
\item Exécution rapide et répétable
\item Détection précoce des régressions
\item Feedback immédiat aux développeurs
\item Réduction des coûts de test manuel
\item Traçabilité et historique des résultats
\end{itemize}
\end{successbox}

\section{Suivi et Contrôle de l'Avancement}

\subsection{Métriques Globales du Projet}

\begin{center}
\renewcommand{\arraystretch}{1.4}
\begin{tabular}{|L{4cm}|L{5cm}|C{2cm}|C{2cm}|}
\hline
\headerrow
\tableheader{Métrique} & \tableheader{Formule / Description} & \tableheader{Objectif} & \tableheader{Source} \
\hline
Nombre total de tests & Count(tests) & 
≥
≥ 100 & Jest + Xray \
\hline
Taux global de réussite & (Réussis / Total) 
×
× 100 & 
≥
≥ 90% & CI/CD \
\hline
Couverture de code & Coverage global & 
≥
≥ 80% & Jest \
\hline
Nombre de défauts & Count(bugs) & Trend 
↓
↓ & Jira \
\hline
Temps pipeline CI/CD & Durée end-to-end & < 10 min & GitHub \
\hline
Taux d'automatisation & (Auto / Total) 
×
× 100 & 
≥
≥ 80% & Manuel \
\hline
\end{tabular}
\end{center}

\subsection{Métriques par Niveau de Test}

\subsubsection{Tests Unitaires (Jest)}

\begin{center}
\renewcommand{\arraystretch}{1.3}
\begin{tabular}{|L{5cm}|C{2.5cm}|L{4.5cm}|}
\hline
\headerrow
\tableheader{Métrique} & \tableheader{Objectif} & \tableheader{Outil de collecte} \
\hline
Nombre de tests unitaires & 
≥
≥ 50 & Jest reporter \
\hline
Taux de réussite unitaire & 
≥
≥ 95% & Jest output \
\hline
Couverture statements & 
≥
≥ 85% & Jest coverage \
\hline
Couverture branches & 
≥
≥ 80% & Jest coverage \
\hline
Couverture functions & 
≥
≥ 90% & Jest coverage \
\hline
Temps d'exécution total & < 30 secondes & Jest --verbose \
\hline
\end{tabular}
\end{center}

\subsubsection{Tests d'Intégration}

\begin{center}
\renewcommand{\arraystretch}{1.3}
\begin{tabular}{|L{5cm}|C{2.5cm}|L{4.5cm}|}
\hline
\headerrow
\tableheader{Métrique} & \tableheader{Objectif} & \tableheader{Outil de collecte} \
\hline
Nombre de scénarios & 
≥
≥ 30 & Supertest reporter \
\hline
Taux de réussite & 
≥
≥ 90% & CI/CD logs \
\hline
Temps moyen réponse API & < 200ms & Supertest response time \
\hline
Défaillances MongoDB & 0 & Error logs \
\hline
Tests flaky (instables) & < 5% & CI/CD history \
\hline
\end{tabular}
\end{center}

\subsubsection{Tests E2E}

\begin{center}
\renewcommand{\arraystretch}{1.3}
\begin{tabular}{|L{5cm}|C{2.5cm}|L{4.5cm}|}
\hline
\headerrow
\tableheader{Métrique} & \tableheader{Objectif} & \tableheader{Outil de collecte} \
\hline
Scénarios E2E exécutés & 
≥
≥ 15 & Newman/Postman reports \
\hline
Taux de réussite E2E & 
≥
≥ 85% & Postman/Chai output \
\hline
Temps moyen parcours & < 2 secondes & Newman timing \
\hline
Défaillances en chaîne & 0 & Analyse manuelle \
\hline
\end{tabular}
\end{center}

\subsubsection{Tests de Performance}

\begin{center}
\renewcommand{\arraystretch}{1.3}
\begin{tabular}{|L{5cm}|C{2.5cm}|L{4.5cm}|}
\hline
\headerrow
\tableheader{Métrique} & \tableheader{Objectif} & \tableheader{Outil de collecte} \
\hline
Temps moyen de réponse (p50) & < 150ms & autocannon/loadtest \
\hline
Temps p95 & < 300ms & autocannon \
\hline
Temps p99 & < 500ms & autocannon \
\hline
Requêtes par seconde & 
≥
≥ 50 req/s & loadtest \
\hline
Taux d'erreurs sous charge & < 1% & loadtest errors \
\hline
Utilisation CPU/Mémoire & < 70% & Monitoring système \
\hline
\end{tabular}
\end{center}

\subsection{Fréquence de Collecte et Reporting}

\begin{center}
\renewcommand{\arraystretch}{1.3}
\begin{tabular}{|L{4.5cm}|C{2.5cm}|L{5cm}|}
\hline
\headerrow
\tableheader{Type de rapport} & \tableheader{Fréquence} & \tableheader{Destinataires} \
\hline
Rapport quotidien CI/CD & Quotidien & Équipe technique \
\hline
Dashboard métriques & Temps réel & Test Lead \
\hline
Rapport hebdomadaire & Hebdomadaire & Management + QA \
\hline
Rapport de couverture & Chaque push & Développeurs \
\hline
Rapport final de clôture & Fin de projet & Toutes parties prenantes \
\hline
\end{tabular}
\end{center}

\section{Gestion de Configuration}

\begin{center}
\renewcommand{\arraystretch}{1.3}
\begin{tabular}{|L{3cm}|L{2.5cm}|L{5cm}|L{3cm}|}
\hline
\headerrow
\tableheader{Élément} & \tableheader{Outil} & \tableheader{Localisation} & \tableheader{Versionnement} \
\hline
Code source & Git / GitHub & Repository principal & Semantic Versioning \
\hline
Tests automatisés & Git / GitHub & /tests/ & Avec le code \
\hline
Pipeline CI/CD & Git / GitHub & .github/workflows/ & Avec le code \
\hline
Cas de test Xray & Jira / Xray & Projet TJAA & Par sprint \
\hline
Documentation & Git / GitHub & /docs/ et README.md & Markdown versionné \
\hline
Données de test & Git / GitHub & /tests/data/ & JSON versionné \
\hline
\end{tabular}
\end{center}

\begin{infobox}[Règles de Gestion]
\begin{itemize}
\item Toute modification du code source nécessite un commit explicite
\item Les tests doivent être mis à jour en même temps que le code
\item Le pipeline CI/CD est déclenché automatiquement à chaque push sur main
\item Les rapports sont archivés pour chaque build
\item Backup quotidien du repository GitHub
\end{itemize}
\end{infobox}

\section{Gestion des Défauts}

\subsection{Cycle de Vie d'un Défaut}

\begin{center}
\begin{tikzpicture}[
node distance=1.8cm,
state/.style={
rectangle, rounded corners=5pt,
minimum width=2.5cm, minimum height=0.8cm,
draw=certilogBlue, line width=1pt,
fill=bgBlue, font=\small\bfseries,
text=certilogDarkBlue
},
arrow/.style={-{Stealth[length=3mm]}, certilogOrange, line width=1.5pt}
]
\node[state] (new) {1. CRÉATION};
\node[state, right=of new] (assigned) {2. AFFECTATION};
\node[state, right=of assigned] (analysis) {3. ANALYSE};
\node[state, below=of analysis] (progress) {4. CORRECTION};
\node[state, left=of progress] (retest) {5. RE-TEST};
\node[state, left=of retest] (verify) {6. VÉRIFICATION};
\node[state, below=of verify, fill=bgGreen] (closed) {7. FERMETURE};
\node[state, right=of closed, fill=bgRed] (reopen) {RÉOUVERTURE};

text

\draw[arrow] (new) -- (assigned);
\draw[arrow] (assigned) -- (analysis);
\draw[arrow] (analysis) -- (progress);
\draw[arrow] (progress) -- (retest);
\draw[arrow] (retest) -- (verify);
\draw[arrow] (verify) -- (closed);
\draw[arrow, dashed] (verify) -- (reopen);
\draw[arrow, dashed] (reopen) to[bend right=30] (progress);
\end{tikzpicture}
\end{center}

\subsection{Criticité des Défauts}

\begin{center}
\renewcommand{\arraystretch}{1.4}
\begin{tabular}{|C{2cm}|L{4.5cm}|L{4cm}|C{2.5cm}|}
\hline
\headerrow
\tableheader{Niveau} & \tableheader{Définition} & \tableheader{Exemple} & \tableheader{Délai} \
\hline
\textcolor{errorRed}{\textbf{Bloquant}} & Empêche toute utilisation & Impossibilité de se connecter & < 24h \
\hline
\textcolor{certilogOrange}{\textbf{Majeur}} & Fonctionnalité critique défaillante & Token non validé & < 48h \
\hline
\textcolor{warningYellow}{\textbf{Mineur}} & Dysfonctionnement non bloquant & Message d'erreur peu clair & < 1 semaine \
\hline
\textcolor{certilogGray}{\textbf{Trivial}} & Problème cosmétique & Faute d'orthographe & Backlog \
\hline
\end{tabular}
\end{center}

\section{Outils de Test Utilisés}

\subsection{Vue d'ensemble des Outils}

\begin{center}
\begin{tikzpicture}[
node distance=2cm and 1.5cm,
tool/.style={
rectangle, rounded corners=8pt,
minimum width=2.8cm, minimum height=1.2cm,
draw=certilogBlue, line width=1pt,
fill=white, font=\small\bfseries,
drop shadow
},
category/.style={
ellipse, minimum width=3cm, minimum height=1cm,
fill=certilogBlue, text=white, font=\bfseries
}
]
% Categories
\node[category] (gestion) at (0,0) {Gestion};
\node[category] (auto) at (6,0) {Automatisation};
\node[category] (qualite) at (12,0) {Qualité};

text

% Gestion tools
\node[tool] (xray) at (-1.5,-2) {Xray};
\node[tool] (jira) at (1.5,-2) {Jira};

% Automatisation tools
\node[tool] (jest) at (4,-2) {Jest};
\node[tool] (supertest) at (6,-2) {Supertest};
\node[tool] (newman) at (8,-2) {Newman};

% Qualité tools
\node[tool] (eslint) at (10.5,-2) {ESLint};
\node[tool] (sonar) at (13.5,-2) {SonarQube};

% CI/CD center
\node[tool, fill=certilogOrange, text=white, minimum width=4cm] (cicd) at (6,-4) {GitHub Actions};

% Connections
\draw[certilogGray, line width=1pt] (gestion) -- (xray);
\draw[certilogGray, line width=1pt] (gestion) -- (jira);
\draw[certilogGray, line width=1pt] (auto) -- (jest);
\draw[certilogGray, line width=1pt] (auto) -- (supertest);
\draw[certilogGray, line width=1pt] (auto) -- (newman);
\draw[certilogGray, line width=1pt] (qualite) -- (eslint);
\draw[certilogGray, line width=1pt] (qualite) -- (sonar);

% To CI/CD
\foreach \tool in {xray, jira, jest, supertest, newman, eslint, sonar} {
    \draw[certilogOrange, line width=1pt, dashed] (\tool) -- (cicd);
}
\end{tikzpicture}
\end{center}

\subsection{Détail des Outils}

\begin{center}
\renewcommand{\arraystretch}{1.3}
\begin{tabular}{|L{2.5cm}|L{2.5cm}|L{4cm}|L{3.5cm}|}
\hline
\headerrow
\tableheader{Outil} & \tableheader{Type} & \tableheader{Utilisation} & \tableheader{Intégration} \
\hline
Xray & Gestion tests & Cas de test, campagnes & Jira \
\hline
Jira & Gestion défauts & Suivi anomalies, workflow & Xray, GitHub \
\hline
Jest & Tests unitaires & Fonctions isolées, couverture & GitHub Actions \
\hline
Supertest & Tests API & Tests d'intégration HTTP & Jest \
\hline
Chai & Assertions & Tests E2E avancés & Postman/Newman \
\hline
Newman & Tests E2E auto & Exécution collections & GitHub Actions \
\hline
ESLint & Analyse statique & Qualité code, conventions & GitHub Actions \
\hline
SonarQube & Qualité code & Code smells, vulnérabilités & Optionnel \
\hline
\end{tabular}
\end{center}

\subsection{Traçabilité Complète}

\begin{center}
\begin{tikzpicture}[
node distance=1.5cm,
step/.style={
rectangle, rounded corners=3pt,
minimum width=3.5cm, minimum height=0.8cm,
fill=#1, text=white, font=\small\bfseries
},
arrow/.style={-{Stealth[length=2mm]}, line width=1.5pt}
]
\node[step=certilogBlue] (req) {1. EXIGENCE (Jira)};
\node[step=certilogTeal, right=of req] (test) {2. CAS DE TEST (Xray)};
\node[step=certilogOrange, right=of test] (script) {3. SCRIPT AUTO};
\node[step=certilogNavy, below=1.2cm of script] (exec) {4. EXÉCUTION CI/CD};
\node[step=successGreen, left=of exec] (result) {5. RÉSULTATS};
\node[step=errorRed, left=of result] (defect) {6. DÉFAUT (si échec)};

text

\draw[arrow, certilogGray] (req) -- (test);
\draw[arrow, certilogGray] (test) -- (script);
\draw[arrow, certilogGray] (script) -- (exec);
\draw[arrow, certilogGray] (exec) -- (result);
\draw[arrow, certilogGray, dashed] (result) -- (defect);
\draw[arrow, certilogGray, dashed] (defect) to[bend right=30] (req);
\end{tikzpicture}
\end{center}

\section{Gestion de la Qualité}

\begin{infobox}[Principes de Qualité Appliqués]
\begin{description}
\item[\faLink\ Traçabilité totale] Chaque exigence a au moins un test associé
\item[\faUsers\ Revue par les pairs] Tout code est revu avant merge
\item[\faCheckDouble\ Critères d'acceptation] Chaque user story a des critères mesurables
\item[\faRedo\ Amélioration continue] Rétrospectives après chaque sprint
\item[\faCode\ Standards de codage] Respect des conventions ESLint
\end{description}
\end{infobox}

\subsection{Audits Qualité Planifiés}

\begin{center}
\renewcommand{\arraystretch}{1.3}
\begin{tabular}{|L{4cm}|C{2.5cm}|L{4cm}|}
\hline
\headerrow
\tableheader{Type d'audit} & \tableheader{Fréquence} & \tableheader{Responsable} \
\hline
Revue de code & Chaque Pull Request & Pair developer \
\hline
Audit des tests & Hebdomadaire & Test Lead \
\hline
Revue de couverture & Chaque push & Automatique (CI/CD) \
\hline
Audit sécurité & Fin de sprint & Test Engineer \
\hline
Validation finale & Avant déploiement & Encadrant \
\hline
\end{tabular}
\end{center}

%==============================================================================
% CHAPTER 7: BESOINS EN ENVIRONNEMENTS
%==============================================================================
\chapter{BESOINS EN ENVIRONNEMENTS}

\begin{center}
\renewcommand{\arraystretch}{1.4}
\begin{tabular}{|C{1.2cm}|L{3cm}|L{4cm}|L{4.5cm}|}
\hline
\headerrow
\tableheader{ID} & \tableheader{Environnement} & \tableheader{Description} & \tableheader{Configuration} \
\hline
ENV-01 & Local Development & Postes de travail & Node.js v18+, MongoDB, npm, Git \
\hline
ENV-02 & CI/CD GitHub Actions & Intégration continue & Ubuntu latest, Node.js v18 \
\hline
ENV-03 & MongoDB Test & Base de données & MongoDB 6.0+, collections test \
\hline
ENV-04 & Staging (optionnel) & Pré-production & Environnement miroir \
\hline
\end{tabular}
\end{center}

\vspace{0.5cm}

\begin{metricbox}[Variables d'Environnement Requises]
\begin{center}
\renewcommand{\arraystretch}{1.3}
\begin{tabular}{|L{3cm}|L{5cm}|L{4.5cm}|}
\hline
\headerrow
\tableheader{Variable} & \tableheader{Description} & \tableheader{Exemple} \
\hline
MONGODB_URI & URL connexion MongoDB & mongodb://localhost:27017/testdb \
\hline
JWT_SECRET & Clé secrète JWT & mySecretKey123 \
\hline
JWT_EXPIRATION & Durée validité token & 86400 (24h) \
\hline
PORT & Port serveur Express & 3000 \
\hline
NODE_ENV & Environnement Node & test / production \
\hline
\end{tabular}
\end{center}
\end{metricbox}

%==============================================================================
% CHAPTER 8: BESOINS EN RESSOURCES ET FORMATION
%==============================================================================
\chapter{BESOINS EN RESSOURCES ET FORMATION}

\section{Ressources Humaines}

\begin{center}
\renewcommand{\arraystretch}{1.4}
\begin{tabular}{|L{2.5cm}|L{3cm}|C{2cm}|C{3cm}|L{3cm}|}
\hline
\headerrow
\tableheader{Rôle} & \tableheader{Nom} & \tableheader{Charge} & \tableheader{Période} & \tableheader{Compétences} \
\hline
Test Lead & Rana ROMDHANE & 50% (4h/j) & 01/10 - 15/12/2025 & Jest, Xray, Jira \
\hline
Test Engineer & Oulimata SALL & 50% (4h/j) & 01/10 - 15/12/2025 & Supertest, CI/CD \
\hline
Encadrant & Hela AOUADI & 10% (conseil) & 01/10 - 15/12/2025 & Validation, revue \
\hline
\end{tabular}
\end{center}

\vspace{0.5cm}
\textbf{Total effort estimé :} 8 jours/personne

\section{Ressources Matérielles}

\begin{center}
\renewcommand{\arraystretch}{1.3}
\begin{tabular}{|L{4cm}|C{2cm}|L{6cm}|}
\hline
\headerrow
\tableheader{Ressource} & \tableheader{Quantité} & \tableheader{Usage} \
\hline
Postes de travail & 2 & Développement et tests locaux \
\hline
Accès GitHub & 2 comptes & Repository et CI/CD \
\hline
Licence Jira/Xray & 2 utilisateurs & Gestion tests et défauts \
\hline
MongoDB Atlas (opt.) & 1 cluster & BD cloud pour tests \
\hline
\end{tabular}
\end{center}

\section{Besoins en Formation}

\begin{center}
\renewcommand{\arraystretch}{1.3}
\begin{tabular}{|L{4cm}|L{3.5cm}|C{1.5cm}|C{2.5cm}|}
\hline
\headerrow
\tableheader{Thème} & \tableheader{Public} & \tableheader{Durée} & \tableheader{Date prévue} \
\hline
Xray / Jira pour les tests & O. SALL & R. ROMDHANE & 2h & 05/10/2025 \
\hline
Jest avancé & O. SALL & R. ROMDHANE & 3h & 10/10/2025 \
\hline
Sécurité API (OWASP) & R. ROMDHANE & 2h & 15/10/2025 \
\hline
GitHub Actions CI/CD & R. ROMDHANE & 2h & 20/10/2025 \
\hline
Techniques de test API & O. SALL & 2h & 12/10/2025 \
\hline
\end{tabular}
\end{center}

\vspace{0.5cm}
\textbf{Total formation :} 11 heures

%==============================================================================
% CHAPTER 9: TÂCHES DE TEST ET RESPONSABILITÉS
%==============================================================================
\chapter{TÂCHES DE TEST ET RESPONSABILITÉS}

\section{Rôles et Responsabilités}

\begin{center}
\renewcommand{\arraystretch}{1.4}
\begin{tabular}{|L{2.5cm}|L{2.5cm}|L{2cm}|L{6cm}|}
\hline
\headerrow
\tableheader{Rôle} & \tableheader{NOM Prénom} & \tableheader{Entité} & \tableheader{Responsabilités principales} \
\hline
Test Lead & ROMDHANE Rana & Équipe QA &
\begin{minipage}[t]{5.5cm}
\vspace{2pt}
\begin{itemize}[leftmargin=1em, topsep=0pt, itemsep=0pt]
\item Planification des tests
\item Création des cas de test
\item Coordination avec l'encadrant
\item Analyse des résultats
\end{itemize}
\vspace{2pt}
\end{minipage} \
\hline
Test Engineer & SALL Oulimata & Équipe QA &
\begin{minipage}[t]{5.5cm}
\vspace{2pt}
\begin{itemize}[leftmargin=1em, topsep=0pt, itemsep=0pt]
\item Automatisation (Jest, Supertest)
\item Configuration CI/CD
\item Tests de sécurité
\item Maintenance du pipeline
\end{itemize}
\vspace{2pt}
\end{minipage} \
\hline
Encadrant & AOUADI Hela & ENICAR &
\begin{minipage}[t]{5.5cm}
\vspace{2pt}
\begin{itemize}[leftmargin=1em, topsep=0pt, itemsep=0pt]
\item Validation du plan de test
\item Revue des livrables
\item Approbation finale
\end{itemize}
\vspace{2pt}
\end{minipage} \
\hline
\end{tabular}
\end{center}

\section{Préparation des Tests}

\begin{center}
\renewcommand{\arraystretch}{1.3}
\footnotesize
\begin{tabular}{|L{4cm}|C{2cm}|C{1cm}|C{2cm}|L{3.5cm}|}
\hline
\headerrow
\tableheader{Description} & \tableheader{Responsable} & \tableheader{Charge} & \tableheader{Date fin} & \tableheader{Livrables} \
\hline
Analyse des exigences & R. ROMDHANE & 0,5 j & 10/10/2025 & Document d'analyse \
\hline
Rédaction cas de test & O. SALL & 1 j & 25/10/2025 & 50+ cas dans Xray \
\hline
Installation outils & Équipe & 0,5 j & 20/10/2025 & Env. opérationnel \
\hline
Configuration CI/CD & O. SALL & 0,5 j & 17/11/2025 & Pipeline fonctionnel \
\hline
Création jeu de données & R. ROMDHANE & 0,5 j & 25/10/2025 & Fichiers JSON \
\hline
Rédaction Plan de Test & O. SALL & 1 j & 01/12/2025 & Ce document \
\hline
\end{tabular}
\end{center}

\vspace{0.3cm}
\textbf{Total préparation :} 4 jours

\section{Exécution des Tests}

\begin{center}
\renewcommand{\arraystretch}{1.3}
\footnotesize
\begin{tabular}{|L{4cm}|C{2cm}|C{1cm}|C{2cm}|L{3.5cm}|}
\hline
\headerrow
\tableheader{Description} & \tableheader{Responsable} & \tableheader{Charge} & \tableheader{Date fin} & \tableheader{Livrables} \\
\hline
Tests unitaires Jest & R. ROMDHANE & 0,5 j & 01/11/2025 & Rapport Jest + couverture \\
\hline
Tests endpoints API & O. SALL & 1 j & 02/11/2025 & Rapport Supertest \\
\hline
Scénarios E2E & R. ROMDHANE & 1 j & 02/11/2025 & Collection Postman \\
\hline
Tests de sécurité & O. SALL & 1 j & 05/11/2025 & Rapport sécurité \\
\hline
Tests performance & R. ROMDHANE & 0,5 j & 05/11/2025 & Rapport performance \\
\hline
Analyse des résultats & O. SALL & 0,5 j & 20/11/2025 & Liste défauts Jira \\
\hline
Correction défauts & Équipe & 1 j & 30/11/2025 & Code corrigé \\
\hline
Re-tests & O. SALL & 0,5 j & 05/12/2025 & Rapport re-test \\
\hline
Rapport Final & O. SALL & 1 j & 15/12/2025 & Rapport de clôture \\
\hline
\end{tabular}
\end{center}

\vspace{0.3cm}
\textbf{Total exécution :} 7 jours

%==============================================================================
% CHAPTER 10: CRITÈRES D'ARRÊT ET CONDITIONS DE REPRISE
%==============================================================================
\chapter{CRITÈRES D'ARRÊT ET CONDITIONS DE REPRISE DES TESTS}

\section{Critères d'Arrêt (Suspension des Tests)}

Les tests seront suspendus immédiatement si l'une des conditions suivantes est remplie :

\begin{center}
\renewcommand{\arraystretch}{1.4}
\begin{tabular}{|L{3.5cm}|L{4.5cm}|L{5cm}|}
\hline
\headerrow
\tableheader{Critère} & \tableheader{Description} & \tableheader{Action corrective} \\
\hline
Taux d'échec critique & Plus de 20\% des tests critiques échouent & Analyse d'impact + correction urgente \\
\hline
Environnement instable & MongoDB indisponible ou CI/CD en panne & Restauration environnement \\
\hline
Bloqueur technique & Bug bloquant empêchant l'exécution & Correction prioritaire du bug \\
\hline
Données corrompues & Impossible de réinitialiser la BD & Recréation des données \\
\hline
Pipeline non fonctionnel & GitHub Actions ne démarre plus & Debug du workflow YAML \\
\hline
Ressources indisponibles & Absence prolongée d'un testeur clé & Réaffectation des tâches \\
\hline
\end{tabular}
\end{center}

\section{Conditions de Reprise des Tests}

Les tests peuvent reprendre uniquement si \textbf{toutes} les conditions suivantes sont satisfaites :

\begin{center}
\renewcommand{\arraystretch}{1.4}
\begin{tabular}{|L{4cm}|L{5.5cm}|C{3cm}|}
\hline
\headerrow
\tableheader{Condition} & \tableheader{Vérification} & \tableheader{Responsable} \\
\hline
Anomalies résolues & Tous les défauts bloquants sont fermés & Test Lead \\
\hline
Environnement validé & MongoDB + CI/CD opérationnels & Test Engineer \\
\hline
Données restaurées & Jeu de données initial recréé et vérifié & Test Engineer \\
\hline
Tests critiques validés & Au moins 90\% des tests critiques passent & Test Lead \\
\hline
Approbation formelle & Encadrant valide la reprise & Encadrant \\
\hline
\end{tabular}
\end{center}

\begin{metricbox}[Processus de Reprise]
\begin{enumerate}
    \item Correction des défauts bloquants
    \item Validation environnement (smoke tests)
    \item Exécution tests critiques uniquement
    \item Si succès > 90\% $\rightarrow$ reprise complète
    \item Sinon $\rightarrow$ nouvelle analyse
\end{enumerate}
\end{metricbox}

%==============================================================================
% CHAPTER 11: LIVRABLES DU TEST
%==============================================================================
\chapter{LIVRABLES DU TEST}

Cette section identifie l'ensemble des livrables produits durant les activités de test.

\section{Cas de Test}

\begin{docinfo}[Description]
Ensemble complet des cas de tests fonctionnels, techniques, d'intégration, de sécurité et E2E créés dans Xray.

\textbf{Contenu :}
\begin{itemize}
    \item Cas de test inscription (signup) - env. 8 scénarios
    \item Cas de test connexion (signin) - env. 10 scénarios
    \item Cas de test gestion des rôles - env. 12 scénarios
    \item Cas de test endpoints protégés - env. 10 scénarios
    \item Cas de test sécurité JWT - env. 8 scénarios
    \item Cas de test performance (basiques) - env. 5 scénarios
\end{itemize}

\textbf{Total estimé :} 50+ cas de test

\textbf{Localisation :}
\begin{itemize}
    \item \faFolder\ Xray – Projet Jira : TJAA
    \item \faFileAlt\ Exemple : [TJAA-35] Campagne de Test - Sprint 1
\end{itemize}

\textbf{Format :} Cas de test structurés Xray avec préconditions, étapes, résultats attendus
\end{docinfo}

\section{Scripts d'Automatisation (Jest / Supertest)}

\begin{docinfo}[Description]
Scripts Node.js permettant l'exécution automatique des tests unitaires et d'intégration.

\textbf{Tests unitaires :}
\begin{itemize}
    \item Validation des données (\texttt{validators.test.js})
    \item Hashing et cryptographie (\texttt{crypto.test.js})
    \item Génération JWT (\texttt{jwt.test.js})
\end{itemize}

\textbf{Tests d'intégration :}
\begin{itemize}
    \item Endpoints d'authentification (\texttt{auth.integration.test.js})
    \item Middleware JWT (\texttt{middleware.test.js})
    \item Accès protégé (\texttt{protected.routes.test.js})
\end{itemize}

\textbf{Localisation :}
\begin{itemize}
    \item \faGithub\ GitHub Repository : \url{https://github.com/RanaRomdhane/node-js-jwt-auth-testing}
    \item \faFolder\ Dossier \texttt{/tests/}
\end{itemize}

\textbf{Format :} Fichiers \texttt{.js} versionnés avec Git
\end{docinfo}

\section{Rapports de Couverture}

\begin{docinfo}[Description]
Rapports générés automatiquement par Jest après chaque exécution CI/CD.

\textbf{Contenu :}
\begin{itemize}
    \item Couverture globale (statements, branches, functions, lines)
    \item Couverture par module/fichier
    \item Fichiers non couverts
    \item Rapport HTML interactif
\end{itemize}

\textbf{Métriques cibles :}
\begin{center}
\begin{tabular}{|l|c|}
\hline
\headerrow
\tableheader{Métrique} & \tableheader{Objectif} \\
\hline
Statements & $\geq$ 80\% \\
\hline
Branches & $\geq$ 80\% \\
\hline
Functions & $\geq$ 80\% \\
\hline
Lines & $\geq$ 80\% \\
\hline
\end{tabular}
\end{center}

\textbf{Localisation :}
\begin{itemize}
    \item GitHub Actions $\rightarrow$ Artifacts $\rightarrow$ coverage-report
    \item Localement : \texttt{/coverage/index.html}
    \item Format JSON : \texttt{/coverage/coverage-final.json}
\end{itemize}

\textbf{Format :} HTML, JSON, LCOV
\end{docinfo}

\section{Rapports de Tests E2E}

\begin{docinfo}[Description]
Rapports issus des tests End-to-End exécutés avec Postman/Newman/Chai.

\textbf{Scénarios E2E couverts :}
\begin{enumerate}
    \item Inscription $\rightarrow$ Connexion $\rightarrow$ Accès ressource user
    \item Inscription admin $\rightarrow$ Connexion $\rightarrow$ Accès ressource admin
    \item Connexion $\rightarrow$ Refresh token $\rightarrow$ Nouvelle requête
    \item Tentative accès sans token $\rightarrow$ Erreur 401
    \item Tentative accès rôle insuffisant $\rightarrow$ Erreur 403
\end{enumerate}

\textbf{Localisation :}
\begin{itemize}
    \item \faArchive\ Workspace Postman (collections exportées)
    \item \faGithub\ GitHub Repository : \texttt{/reports/e2e/}
    \item \faJira\ Xray : Campagnes d'exécution E2E
\end{itemize}

\textbf{Format :} HTML, JSON
\end{docinfo}

\section{Rapport de Performance}

\begin{docinfo}[Description]
Résultats des tests de charge effectués avec autocannon / loadtest.

\textbf{Métriques principales :}
\begin{itemize}
    \item Transactions par seconde (TPS)
    \item Latence moyenne, p95, p99
    \item Taux d'erreur (HTTP 4xx, 5xx)
    \item Débit (throughput) en req/s
\end{itemize}

\textbf{Objectifs de performance :}
\begin{center}
\begin{tabular}{|l|c|}
\hline
\headerrow
\tableheader{Métrique} & \tableheader{Objectif} \\
\hline
Temps de réponse moyen & < 150ms \\
\hline
p95 & < 300ms \\
\hline
p99 & < 500ms \\
\hline
Taux d'erreur & < 1\% \\
\hline
Support charge & 50 req/s \\
\hline
\end{tabular}
\end{center}

\textbf{Format :} TXT, JSON, graphiques PNG
\end{docinfo}

\section{Rapport d'Anomalies}

\begin{docinfo}[Description]
Liste exhaustive des défauts détectés pendant les phases de test.

\textbf{Structure du rapport :}
\begin{enumerate}
    \item Résumé exécutif (nombre total, répartition par criticité)
    \item Défauts bloquants (liste détaillée)
    \item Défauts majeurs (liste résumée)
    \item Défauts mineurs/triviaux (liste)
    \item Défauts reportés (backlog futur)
    \item Tendances et recommandations
\end{enumerate}

\textbf{Localisation :} \faJira\ Jira – Projet TJAA

\textbf{Format :} Jira (online), Excel (export)
\end{docinfo}

\section{Rapport Final de Test (Rapport de Clôture)}

\begin{docinfo}[Description]
Synthèse finale des activités de test conformément au modèle Certilog.

\textbf{Contenu :}
\begin{itemize}
    \item \textbf{Résumé exécutif} : Objectifs atteints, taux de réussite, défis rencontrés
    \item \textbf{Résultats globaux} : Nombre de tests, taux de réussite, couverture, défauts
    \item \textbf{Analyse qualité} : Conformité fonctionnelle, sécurité, performance
    \item \textbf{Métriques CI/CD} : Stabilité pipeline, temps d'exécution, automatisation
    \item \textbf{Défauts restants} : Défauts ouverts, risques acceptés, plan futur
    \item \textbf{Recommandations} : Améliorations, tests à ajouter, optimisations
    \item \textbf{Leçons apprises} : Bonnes pratiques, difficultés rencontrées
    \item \textbf{Conclusion} : Avis final, recommandation de mise en production
\end{itemize}

\textbf{Localisation :}
\begin{itemize}
    \item \faFileWord\ Document Word : \texttt{Rapport\_de\_cloture\_v1.0.docx}
    \item \faFilePdf\ GitHub Repository : \texttt{/documents/Rapport\_Final\_Test.pdf}
\end{itemize}

\textbf{Date de livraison :} 15/12/2025
\end{docinfo}

\section{Données de Test}

\begin{docinfo}[Description]
Jeux de données MongoDB utilisés pour les tests automatisés.

\textbf{Utilisateurs de test :}
\begin{codebox}[JSON]
\begin{lstlisting}[language=Java]
{
  "username": "testuser",
  "email": "test@example.com",
  "password": "hashed_password",
  "roles": ["user"]
}
\end{lstlisting}
\end{codebox}

\textbf{Rôles disponibles :}
\begin{itemize}
    \item \texttt{user} - accès basique
    \item \texttt{moderator} - accès intermédiaire
    \item \texttt{admin} - accès total
\end{itemize}

\textbf{Tokens de test :}
\begin{itemize}
    \item Tokens valides avec différentes expirations
    \item Tokens expirés
    \item Tokens mal formés (pour tests sécurité)
\end{itemize}

\textbf{Localisation :}
\begin{itemize}
    \item \faGithub\ GitHub Repository : \texttt{/tests/data/}
    \item \faCode\ Script d'initialisation : \texttt{/tests/setup/seed-test-db.js}
\end{itemize}

\textbf{Format :} JSON
\end{docinfo}

\section{Outils de Support}

\begin{docinfo}[Description]
Outils et scripts utilisés pour faciliter et exécuter les tests.

\textbf{Mocks et Stubs Jest :}
\begin{itemize}
    \item Mock MongoDB : \texttt{\_\_mocks\_\_/mongoose.js}
    \item Mock JWT : \texttt{\_\_mocks\_\_/jsonwebtoken.js}
    \item Stub bcrypt : \texttt{\_\_mocks\_\_/bcryptjs.js}
\end{itemize}

\textbf{Configuration CI/CD :}
\begin{itemize}
    \item \texttt{.github/workflows/main.yml} (pipeline principal)
\end{itemize}

\textbf{Scripts utilitaires :}
\begin{itemize}
    \item \texttt{tests/utils/db-helper.js} (connexion/déconnexion DB)
    \item \texttt{tests/utils/token-generator.js} (génération tokens)
    \item \texttt{tests/utils/reset-db.js} (réinitialisation DB)
\end{itemize}

\textbf{Collections Postman :}
\begin{itemize}
    \item \texttt{postman/JWT-Auth-API.postman\_collection.json}
    \item \texttt{postman/test-environment.json}
\end{itemize}

\textbf{Format :} JavaScript, JSON, YAML
\end{docinfo}

%==============================================================================
% CHAPTER 12: RISQUES ET CONTINGENCES
%==============================================================================
\chapter{RISQUES ET CONTINGENCES}

\begin{landscape}
\begin{center}
\renewcommand{\arraystretch}{1.3}
\footnotesize
\begin{longtable}{|C{1.2cm}|L{3.5cm}|C{1.2cm}|C{1.5cm}|L{5.5cm}|C{2cm}|}
\hline
\headerrow
\tableheader{ID} & \tableheader{Description du risque} & \tableheader{Impact} & \tableheader{Prob.} & \tableheader{Contingence} & \tableheader{Responsable} \\
\hline
\endfirsthead
\hline
\headerrow
\tableheader{ID} & \tableheader{Description du risque} & \tableheader{Impact} & \tableheader{Prob.} & \tableheader{Contingence} & \tableheader{Responsable} \\
\hline
\endhead

RISK-01 & Instabilité de MongoDB (connexions perdues, timeouts) & Fort & \probhigh & 
\begin{minipage}[t]{5cm}
\vspace{2pt}
\begin{itemize}[leftmargin=0.8em, topsep=0pt, itemsep=0pt, parsep=0pt]
    \item MongoDB Memory Server pour tests
    \item Script de réinitialisation automatique
    \item Retry logic dans les tests
\end{itemize}
\vspace{2pt}
\end{minipage} & R. ROMDHANE \\
\hline

RISK-02 & Token JWT expiré pendant tests (longues exécutions) & Moyen & \probmedium & 
\begin{minipage}[t]{5cm}
\vspace{2pt}
\begin{itemize}[leftmargin=0.8em, topsep=0pt, itemsep=0pt, parsep=0pt]
    \item Nouveaux tokens avant chaque suite
    \item Tokens avec expiration longue (24h)
    \item Refresh automatique dans tests E2E
\end{itemize}
\vspace{2pt}
\end{minipage} & O. SALL \\
\hline

RISK-03 & Pipeline CI/CD en panne (GitHub Actions indisponible) & Fort & \problow & 
\begin{minipage}[t]{5cm}
\vspace{2pt}
\begin{itemize}[leftmargin=0.8em, topsep=0pt, itemsep=0pt, parsep=0pt]
    \item Exécution locale possible (npm test)
    \item Documentation des commandes manuelles
    \item Rapports locaux exportables vers Xray
\end{itemize}
\vspace{2pt}
\end{minipage} & R. ROMDHANE \\
\hline

RISK-04 & Défauts bloquants multiples (> 20\% échecs critiques) & Fort & \probmedium & 
\begin{minipage}[t]{5cm}
\vspace{2pt}
\begin{itemize}[leftmargin=0.8em, topsep=0pt, itemsep=0pt, parsep=0pt]
    \item Suspension immédiate des tests
    \item Session de debug prioritaire
    \item Réaffectation des ressources
    \item Extension délai si nécessaire
\end{itemize}
\vspace{2pt}
\end{minipage} & Test Lead \\
\hline

RISK-05 & Indisponibilité d'un testeur (maladie, urgence) & Moyen & \probmedium & 
\begin{minipage}[t]{5cm}
\vspace{2pt}
\begin{itemize}[leftmargin=0.8em, topsep=0pt, itemsep=0pt, parsep=0pt]
    \item Cross-training entre testeurs
    \item Documentation détaillée des tâches
    \item Backup plan avec l'encadrant
    \item Priorisation tests critiques
\end{itemize}
\vspace{2pt}
\end{minipage} & Test Lead \\
\hline

RISK-06 & Données de test corrompues (modifications accidentelles) & Moyen & \probmedium & 
\begin{minipage}[t]{5cm}
\vspace{2pt}
\begin{itemize}[leftmargin=0.8em, topsep=0pt, itemsep=0pt, parsep=0pt]
    \item Versioning des données dans Git
    \item Script de reset automatique
    \item Backup quotidien des données
\end{itemize}
\vspace{2pt}
\end{minipage} & R. ROMDHANE \\
\hline

RISK-07 & Couverture de code insuffisante (< 80\%) & Moyen & \probmedium & 
\begin{minipage}[t]{5cm}
\vspace{2pt}
\begin{itemize}[leftmargin=0.8em, topsep=0pt, itemsep=0pt, parsep=0pt]
    \item Sprint dédié tests unitaires
    \item Analyse des zones non couvertes
    \item Priorisation du code critique
\end{itemize}
\vspace{2pt}
\end{minipage} & O. SALL \\
\hline

RISK-08 & Problèmes de compatibilité (versions Node.js, dépendances) & Faible & \problow & 
\begin{minipage}[t]{5cm}
\vspace{2pt}
\begin{itemize}[leftmargin=0.8em, topsep=0pt, itemsep=0pt, parsep=0pt]
    \item Lock des versions dans package-lock.json
    \item Tests de compatibilité pré-intégration
    \item Utilisation de nvm
\end{itemize}
\vspace{2pt}
\end{minipage} & R. ROMDHANE \\
\hline

\end{longtable}
\end{center}
\end{landscape}

\begin{infobox}[Légende Probabilité]
\begin{itemize}
    \item \textcolor{errorRed}{\faCircle} \textbf{Élevée} (> 50\%)
    \item \textcolor{warningYellow}{\faCircle} \textbf{Moyenne} (20-50\%)
    \item \textcolor{successGreen}{\faCircle} \textbf{Faible} (< 20\%)
\end{itemize}
\end{infobox}

%==============================================================================
% CHAPTER 13: CRITÈRES DE PASSAGE OU ÉCHEC
%==============================================================================
\chapter{CRITÈRES DE PASSAGE OU ÉCHEC}

\section{Critères de Succès (GO)}

Le projet de test est considéré comme \textbf{RÉUSSI} si \textbf{TOUS} les critères suivants sont satisfaits :

\begin{successbox}[Critères de Succès]
\begin{center}
\renewcommand{\arraystretch}{1.4}
\begin{tabular}{|L{4.5cm}|C{2.5cm}|L{5.5cm}|}
\hline
\headerrow
\tableheader{Critère} & \tableheader{Seuil minimum} & \tableheader{Mesure} \\
\hline
Taux de réussite des tests & $\geq$ 90\% & (Tests réussis / Total) $\times$ 100 \\
\hline
Couverture de code & $\geq$ 80\% & Jest coverage (statements, branches, functions) \\
\hline
Défauts critiques & 0 défaut ouvert & Jira - criticité = Bloquant \\
\hline
Défauts majeurs & $\leq$ 2 défauts & Jira - criticité = Majeur (avec justification) \\
\hline
Pipeline CI/CD & $\geq$ 95\% succès & GitHub Actions - 10 derniers runs \\
\hline
Tests de sécurité & 100\% validés & Tous scénarios OWASP passés \\
\hline
Tests de performance & Objectifs atteints & Temps < 300ms (p95), erreur < 1\% \\
\hline
Documentation livrée & 100\% & Tous les livrables présents et validés \\
\hline
\end{tabular}
\end{center}

\vspace{0.5cm}
\begin{center}
\begin{tikzpicture}
    \node[fill=successGreen, text=white, rounded corners=5pt, 
          inner xsep=20pt, inner ysep=10pt, font=\Large\bfseries] {
        \faCheckCircle\quad MISE EN PRODUCTION AUTORISÉE
    };
\end{tikzpicture}
\end{center}
\end{successbox}

\section{Critères d'Échec (NO-GO)}

Le projet de test est considéré comme \textbf{ÉCHOUÉ} si \textbf{AU MOINS UN} des critères suivants est présent :

\begin{errorbox}[Critères d'Échec]
\begin{center}
\renewcommand{\arraystretch}{1.4}
\begin{tabular}{|L{4cm}|C{2.5cm}|L{5.5cm}|}
\hline
\headerrow
\tableheader{Critère bloquant} & \tableheader{Seuil d'alerte} & \tableheader{Impact} \\
\hline
Taux de réussite & < 70\% & Qualité insuffisante, risque élevé \\
\hline
Défaut critique ouvert & $\geq$ 1 défaut & Fonctionnalité majeure défaillante \\
\hline
Couverture de code & < 60\% & Zones non testées trop importantes \\
\hline
Pipeline CI/CD & Échec constant & Qualité en continu impossible \\
\hline
Tests sécurité échoués & $\geq$ 1 vulnérabilité & Risque de sécurité inacceptable \\
\hline
Performance dégradée & Temps > 1s (p95) & Expérience utilisateur inacceptable \\
\hline
\end{tabular}
\end{center}

\vspace{0.5cm}
\begin{center}
\begin{tikzpicture}
    \node[fill=errorRed, text=white, rounded corners=5pt, 
          inner xsep=20pt, inner ysep=10pt, font=\Large\bfseries] {
        \faTimesCircle\quad MISE EN PRODUCTION REFUSÉE
    };
\end{tikzpicture}
\end{center}
\end{errorbox}

\section{Critères d'Acceptation Conditionnelle}

Si les critères suivants sont présents, une \textbf{mise en production conditionnelle} peut être envisagée :

\begin{notebox}[Acceptation Conditionnelle]
\begin{center}
\renewcommand{\arraystretch}{1.4}
\begin{tabular}{|L{3cm}|C{3cm}|L{6cm}|}
\hline
\headerrow
\tableheader{Situation} & \tableheader{Condition} & \tableheader{Plan d'action requis} \\
\hline
Taux de réussite & Entre 70\% et 90\% & Analyse détaillée + correction sous 48h \\
\hline
Défauts majeurs & Entre 3 et 5 défauts & Plan de correction priorisé + hotfix prévu \\
\hline
Couverture de code & Entre 60\% et 80\% & Sprint dédié tests + monitoring renforcé \\
\hline
Performance & p95 entre 300-500ms & Optimisation planifiée + monitoring \\
\hline
\end{tabular}
\end{center}

\vspace{0.5cm}
\begin{center}
\begin{tikzpicture}
    \node[fill=warningYellow, text=certilogDarkGray, rounded corners=5pt, 
          inner xsep=20pt, inner ysep=10pt, font=\Large\bfseries] {
        \faExclamationTriangle\quad MISE EN PRODUCTION SOUS RÉSERVE
    };
\end{tikzpicture}
\end{center}
\end{notebox}

\section{Processus de Validation Finale}

\begin{center}
\begin{tikzpicture}[
    node distance=1.5cm,
    step/.style={
        rectangle, rounded corners=5pt,
        minimum width=4cm, minimum height=1cm,
        fill=certilogBlue, text=white, font=\small\bfseries,
        drop shadow
    },
    arrow/.style={-{Stealth[length=3mm]}, certilogOrange, line width=2pt}
]
    \node[step] (step1) {1. Auto-évaluation (Test Lead)};
    \node[step, below=of step1] (step2) {2. Revue par les pairs};
    \node[step, below=of step2] (step3) {3. Présentation encadrant};
    \node[step, below=of step3] (step4) {4. Décision formelle};
    \node[step, below=of step4, fill=successGreen] (step5) {5. Documentation};
    
    \draw[arrow] (step1) -- (step2);
    \draw[arrow] (step2) -- (step3);
    \draw[arrow] (step3) -- (step4);
    \draw[arrow] (step4) -- (step5);
    
    % Decision labels
    \node[right=1cm of step4, font=\small] {GO / NO-GO / Conditionnel};
\end{tikzpicture}
\end{center}

\vspace{0.5cm}
\begin{center}
\fcolorbox{certilogBlue}{bgBlue}{
    \parbox{0.8\textwidth}{
        \centering
        \textbf{\faCalendarCheck\ Date de validation finale :} 15/12/2025
    }
}
\end{center}

%==============================================================================
% CHAPTER 14: INFORMATIONS COMPLÉMENTAIRES
%==============================================================================
\chapter{INFORMATIONS COMPLÉMENTAIRES}

\section{Contexte Académique}

Ce projet s'inscrit dans le cadre d'un \textbf{projet académique ENICAR} visant à :

\begin{itemize}
    \item Appliquer les méthodologies de test professionnelles
    \item Maîtriser les outils modernes de test et CI/CD
    \item Développer des compétences en assurance qualité logicielle
    \item Préparer à des certifications de test (ISTQB)
\end{itemize}

\begin{infobox}[Encadrement]
Le projet est supervisé par \textbf{Mme Hela AOUADI}, enseignante à l'ENICAR, spécialiste en génie logiciel et qualité.
\end{infobox}

\section{Contraintes Spécifiques}

\begin{center}
\renewcommand{\arraystretch}{1.4}
\begin{tabular}{|L{3.5cm}|L{9.5cm}|}
\hline
\headerrow
\tableheader{Type de contrainte} & \tableheader{Description} \\
\hline
Réglementaire & Aucune contrainte légale particulière (projet académique) \\
\hline
Budgétaire & Budget limité : utilisation d'outils gratuits/open-source uniquement \\
\hline
Temporelle & Deadline stricte : 15/12/2025 (fin du semestre) \\
\hline
Technologique & Stack : Node.js, MongoDB, JWT, GitHub Actions \\
\hline
Pédagogique & Respect du template Certilog + présentation orale obligatoire \\
\hline
\end{tabular}
\end{center}

\section{Spécificités Techniques}

\subsection{Architecture du Projet}

\begin{codebox}[Structure du projet]
\begin{lstlisting}[language=bash, basicstyle=\ttfamily\small\color{white}]
node-js-jwt-auth-testing/
|-- src/
|   |-- controllers/    (logique metier)
|   |-- models/         (schemas MongoDB)
|   |-- middlewares/    (auth, validation)
|   |-- routes/         (endpoints API)
|   +-- utils/          (helpers)
|-- tests/
|   |-- unit/           (tests Jest unitaires)
|   |-- integration/    (tests Supertest)
|   |-- e2e/            (tests Chai/Newman)
|   |-- data/           (fixtures)
|   +-- utils/          (test helpers)
|-- .github/
|   +-- workflows/      (CI/CD)
+-- coverage/           (rapports)
\end{lstlisting}
\end{codebox}

\subsection{Technologies Utilisées}

\begin{center}
\begin{tikzpicture}[
    tech/.style={
        rectangle, rounded corners=5pt,
        minimum width=3.5cm, minimum height=0.8cm,
        fill=#1, text=white, font=\small\bfseries
    }
]
    % Backend
    \node[tech=certilogBlue] (node) at (0,0) {\faNodeJs\ Node.js v18+};
    \node[tech=certilogTeal] (express) at (4,0) {Express v4};
    \node[tech=successGreen] (mongo) at (8,0) {\faDatabase\ MongoDB 6.0+};
    
    % Auth
    \node[tech=certilogOrange] (jwt) at (0,-1.2) {\faKey\ JWT};
    \node[tech=certilogNavy] (bcrypt) at (4,-1.2) {\faLock\ bcryptjs};
    
    % Tests
    \node[tech=errorRed] (jest) at (0,-2.4) {Jest};
    \node[tech=warningYellow, text=certilogDarkGray] (supertest) at (4,-2.4) {Supertest};
    \node[tech=certilogGray] (chai) at (8,-2.4) {Chai};
    
    % CI/CD
    \node[tech=certilogDarkBlue] (github) at (4,-3.6) {\faGithub\ GitHub Actions};
\end{tikzpicture}
\end{center}

\section{Points d'Attention Particuliers}

\begin{metricbox}[Points Critiques]
\begin{description}[leftmargin=1.5cm, labelwidth=1.3cm]
    \item[\faShieldAlt\ Sécurité JWT] Validation rigoureuse des tokens (expiration, signature, payload) - tests de sécurité prioritaires
    \item[\faUserShield\ Gestion des rôles] Vérification stricte des permissions - tests d'autorisation exhaustifs
    \item[\faTachometerAlt\ Performance MongoDB] Indexation correcte des collections - tests de performance sur requêtes
    \item[\faLink\ Traçabilité] Chaque exigence $\rightarrow$ test $\rightarrow$ résultat $\rightarrow$ défaut doit être lié dans Xray
\end{description}
\end{metricbox}

\section{Livrables Finaux du Projet Global}

En plus des livrables de test, le projet académique nécessite :

\begin{checklist}
    \item Code source complet (GitHub)
    \item Documentation technique (README, API docs)
    \item Plan de Test (ce document)
    \item Rapport Final de Test
    \item Présentation PowerPoint (soutenance)
    \item Démonstration live (API fonctionnelle + tests)
\end{checklist}

\begin{center}
\fcolorbox{certilogOrange}{bgOrange}{
    \parbox{0.6\textwidth}{
        \centering
        \textbf{\faCalendarAlt\ Date de soutenance prévue :} 20/12/2025
    }
}
\end{center}

\section{Critères d'Évaluation Académique}

\begin{center}
\begin{tikzpicture}
    % Pie chart style evaluation criteria
    \foreach \p/\c/\l [count=\i from 0] in {
        15/certilogBlue/Qualité du Plan de Test,
        20/certilogTeal/Couverture des tests,
        20/certilogOrange/Automatisation et CI/CD,
        15/successGreen/Qualité du code,
        15/certilogNavy/Rapport Final de Test,
        15/errorRed/Présentation et démo
    } {
        \pgfmathsetmacro{\startangle}{90 - \i*60}
        \pgfmathsetmacro{\endangle}{\startangle - \p*3.6}
    }
    
    % Table instead of pie chart for clarity
    \node at (0,0) {
        \begin{tabular}{|L{5.5cm}|C{2.5cm}|}
        \hline
        \headerrow
        \tableheader{Critère} & \tableheader{Pondération} \\
        \hline
        Qualité du Plan de Test & 15\% \\
        \hline
        Couverture des tests & 20\% \\
        \hline
        Automatisation et CI/CD & 20\% \\
        \hline
        Qualité du code (ESLint, best practices) & 15\% \\
        \hline
        Rapport Final de Test & 15\% \\
        \hline
        Présentation et démonstration & 15\% \\
        \hline
        \textbf{TOTAL} & \textbf{100\%} \\
        \hline
        \end{tabular}
    };
\end{tikzpicture}
\end{center}

\section{Contacts et Support}

\begin{docinfo}[Support Technique]
\begin{itemize}
    \item \faGithub\ \textbf{GitHub Issues :} \url{https://github.com/RanaRomdhane/node-js-jwt-auth-testing/issues}
    \item \faEnvelope\ \textbf{Email équipe :} 
    \begin{itemize}
        \item oulimata.sall@enicar.ucar.tn
        \item rana.romdhane@enicar.ucar.tn
    \end{itemize}
\end{itemize}
\end{docinfo}

\begin{docinfo}[Support Pédagogique]
\begin{itemize}
    \item \faChalkboardTeacher\ \textbf{Encadrante :} hela.boukhriss@enicar.ucar.tn
\end{itemize}
\end{docinfo}

%==============================================================================
% CHAPTER 15: GLOSSAIRE
%==============================================================================
\chapter{GLOSSAIRE}

\begin{center}
\renewcommand{\arraystretch}{1.3}
\footnotesize
\begin{longtable}{|L{3cm}|L{10cm}|}
\hline
\headerrow
\tableheader{Terme} & \tableheader{Définition} \\
\hline
\endfirsthead
\hline
\headerrow
\tableheader{Terme} & \tableheader{Définition} \\
\hline
\endhead

API & Application Programming Interface - interface permettant la communication entre systèmes logiciels \\
\hline
Artifact & Fichier produit par un pipeline CI/CD (rapports, logs, binaires) \\
\hline
Bcrypt & Algorithme de hachage cryptographique pour sécuriser les mots de passe \\
\hline
Branch & Branche conditionnelle dans le code (if/else) \\
\hline
CI/CD & Continuous Integration / Continuous Deployment - pratiques d'automatisation du développement \\
\hline
Coverage & Couverture de code - pourcentage de code exécuté par les tests \\
\hline
E2E & End-to-End - tests de bout en bout simulant le parcours utilisateur complet \\
\hline
ESLint & Outil d'analyse statique de code JavaScript \\
\hline
Flaky test & Test instable donnant des résultats aléatoires (succès/échec) \\
\hline
GitHub Actions & Plateforme CI/CD intégrée à GitHub \\
\hline
Jest & Framework de test JavaScript développé par Facebook \\
\hline
Jira & Outil de gestion de projet et de suivi des défauts \\
\hline
JWT & JSON Web Token - standard d'authentification basé sur des tokens \\
\hline
Lint / Linting & Analyse statique du code pour détecter erreurs et violations de conventions \\
\hline
Mock & Objet simulé remplaçant une dépendance réelle dans les tests \\
\hline
MongoDB & Base de données NoSQL orientée documents \\
\hline
Newman & Outil CLI pour exécuter les collections Postman \\
\hline
Node.js & Environnement d'exécution JavaScript côté serveur \\
\hline
OWASP & Open Web Application Security Project - référence en sécurité web \\
\hline
p95 / p99 & Percentile 95/99 - mesure de latence (95\% ou 99\% des requêtes sous ce temps) \\
\hline
Pipeline & Chaîne automatisée d'étapes CI/CD (build, test, deploy) \\
\hline
Postman & Outil de test d'API avec interface graphique \\
\hline
RACI & Responsible, Accountable, Consulted, Informed - matrice de responsabilités \\
\hline
REST & Representational State Transfer - architecture d'API web \\
\hline
Smoke test & Test rapide vérifiant les fonctionnalités de base d'une application \\
\hline
SonarQube & Plateforme d'analyse de qualité de code \\
\hline
Sprint & Période de développement itérative (généralement 1-4 semaines) \\
\hline
Stub & Version simplifiée d'un composant pour les tests \\
\hline
Supertest & Bibliothèque Node.js pour tester les API HTTP \\
\hline
TDD & Test-Driven Development - développement guidé par les tests \\
\hline
TPS & Transactions Per Second - nombre de transactions par seconde \\
\hline
Xray & Extension Jira pour la gestion avancée des tests \\
\hline

\end{longtable}
\end{center}

%==============================================================================
% SIGNATURES ET APPROBATIONS
%==============================================================================
\chapter*{SIGNATURES ET APPROBATIONS}
\addcontentsline{toc}{chapter}{SIGNATURES ET APPROBATIONS}

\vspace{1cm}

\begin{center}
\begin{tikzpicture}
    \node[fill=certilogBlue, text=white, minimum width=14cm, minimum height=1.2cm, 
          rounded corners=5pt, font=\Large\bfseries] {
        VALIDATION DU DOCUMENT
    };
\end{tikzpicture}
\end{center}

\vspace{1.5cm}

\begin{center}
\renewcommand{\arraystretch}{2.5}
\begin{tabular}{|L{3cm}|L{4cm}|C{4cm}|C{2.5cm}|}
\hline
\headerrow
\tableheader{Rôle} & \tableheader{Nom} & \tableheader{Signature} & \tableheader{Date} \\
\hline
\textbf{Rédigé par} & 
\begin{minipage}[t]{3.5cm}
Rana ROMDHANE\\
\&\\
Oulimata SALL
\end{minipage} & 
& 01/12/2025 \\
\hline
\textbf{Vérifié par} & Rana ROMDHANE & & 01/12/2025 \\
\hline
\textbf{Approuvé par} & Hela AOUADI & & \_\_\_ / \_\_\_ / 2025 \\
\hline
\end{tabular}
\end{center}

\vspace{2cm}

\begin{center}
\begin{tikzpicture}
    % Official stamp placeholder
    \draw[certilogBlue, line width=2pt, dashed] (0,0) circle (2cm);
    \node[text=certilogGray, font=\small] at (0,0) {Cachet officiel};
\end{tikzpicture}
\end{center}

\vspace{2cm}

\begin{center}
\fcolorbox{certilogBlue}{bgBlue}{
    \parbox{0.9\textwidth}{
        \centering
        \vspace{0.5cm}
        \textbf{\Large Document conforme au template Certilog v1.0.1}
        \vspace{0.3cm}
        
        \faUniversity\ ENICAR - École Nationale d'Ingénieurs de Carthage
        
        \vspace{0.3cm}
        \faCalendarAlt\ Année académique 2025-2026
        \vspace{0.5cm}
    }
}
\end{center}

%==============================================================================
% BACK COVER
%==============================================================================
\newpage
\thispagestyle{empty}

\begin{tikzpicture}[remember picture, overlay]
    % Background
    \fill[certilogDarkBlue] (current page.south west) rectangle (current page.north east);
    
    % Decorative elements
    \foreach \x/\y/\r/\o in {3/5/4/0.1, 15/-3/5/0.08, -2/20/3/0.12} {
        \fill[white, opacity=\o] ([xshift=\x cm, yshift=\y cm]current page.south west) 
            circle (\r cm);
    }
    
    % Grid pattern
    \foreach \i in {0,1,...,21} {
        \draw[white, opacity=0.03, line width=0.5pt] 
            ([xshift=\i cm]current page.south west) -- 
            ([xshift=\i cm]current page.north west);
    }
    \foreach \i in {0,1,...,29} {
        \draw[white, opacity=0.03, line width=0.5pt] 
            ([yshift=\i cm]current page.south west) -- 
            ([yshift=\i cm]current page.south east);
    }
    
    % Content
    \node[anchor=center] at (current page.center) {
        \begin{minipage}{12cm}
            \centering
            
            % Logo
            \begin{tikzpicture}
                \node[fill=white, rounded corners=10pt, inner sep=15pt] {
                    \begin{tabular}{c}
                        \textcolor{certilogBlue}{\fontsize{30}{36}\selectfont\bfseries CERTILOG}\\[-3pt]
                        \textcolor{certilogOrange}{\small\itshape Quality Assurance Template}
                    \end{tabular}
                };
            \end{tikzpicture}
            
            \vspace{2cm}
            
            % Title
            {\color{white}\fontsize{24}{30}\selectfont\bfseries
            Plan de Test\\[10pt]
            Node.js JWT Authentication}
            
            \vspace{1.5cm}
            
            % Divider
            {\color{certilogOrange}\rule{6cm}{3pt}}
            
            \vspace{1.5cm}
            
            % Project info
            {\color{white}\Large
            Référence: \textbf{TJAA-PT-001}\\[8pt]
            Version: \textbf{1.0}\\[8pt]
            Date: \textbf{01/12/2025}}
            
            \vspace{2cm}
            
            % Authors
            {\color{certilogLightGray}\large
            Préparé par\\[10pt]
            \textbf{\color{white}Oulimata SALL \& Rana ROMDHANE}\\[5pt]
            Équipe QA - ENICAR}
            
            \vspace{1.5cm}
            
            % Footer
            {\color{certilogMediumGray}\small
            \faGithub\ github.com/RanaRomdhane/node-js-jwt-auth-testing\\[5pt]
            \faEnvelope\ enicar-team-testing@enicar.ucar.tn}
            
        \end{minipage}
    };
    
    % Copyright footer
    \node[anchor=south, text=certilogMediumGray, font=\small] 
        at ([yshift=1.5cm]current page.south) {
        \textcopyright\ 2025 ENICAR - Tous droits réservés | Document confidentiel
    };
    
\end{tikzpicture}

\end{document}